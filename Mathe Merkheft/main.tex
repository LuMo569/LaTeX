\documentclass[12pt,a4paper]{report}
\renewcommand{\baselinestretch}{1.2} 
\usepackage{graphicx} % Required for inserting images
\usepackage{amsmath}
\usepackage{tikz}
\usetikzlibrary{shapes.geometric, arrows}
\usepackage{pgfplots}
\pgfplotsset{compat=newest}
\usepackage[margin=1in]{geometry}
\usepackage{amssymb}
\usepackage{adjustbox}
\usepackage{subcaption}
\usepackage{tabularx}
\usepackage{caption}
\usepackage{gauss}
\usepackage{pst-eucl}
\usepackage{esvect}
\usepackage{tikz-3dplot}
\usepackage{array}
\usepackage{outlines}
\usepackage[fleqn]{mathtools}
\usetikzlibrary{angles, quotes}

\setlength{\abovedisplayskip}{2pt}
\setlength{\belowdisplayskip}{0pt}
\setlength{\abovedisplayshortskip}{0pt}
\setlength{\belowdisplayshortskip}{0pt}

%Vektoren
\usepackage{physics}
\usepackage[outline]{contour} % glow around text
\usetikzlibrary{angles,quotes} % for pic
\tikzset{>=latex}
\tikzstyle{vector}=[->,very thick]

%Ebenen
\usetikzlibrary{calc}
\usepackage{physics}

\usetikzlibrary{decorations.markings}
\tikzset{->-/.style={decoration={% https://tex.stackexchange.com/a/39282/194703
  markings,
  mark=at position #1 with {\arrow{>}}},postaction={decorate}},
  ->-/.default=0.5}


\usepackage{etoolbox}
\makeatletter
\patchcmd\g@matrix
 {\vbox\bgroup}
 {\vbox\bgroup\normalbaselines}% restore the standard baselineskip
 {}{}
\makeatother
\usetikzlibrary{shapes.misc}

\tikzset{cross/.style={cross out, draw=black, minimum size=2*(#1-\pgflinewidth), inner sep=0pt, outer sep=0pt},cross/.default={2.5pt}}


\newcommand{\BAR}{%
  \hspace{-\arraycolsep}%
  \strut\vrule % the `\vrule` is as high and deep as a strut
  \hspace{-\arraycolsep}%
}

\tikzstyle{process} = [rectangle, minimum width=3cm, minimum height=1cm, text centered, draw=black, fill=lightgray!20]
\tikzstyle{arrow} = [thick,->,>=stealth]

%\captionsetup{justification=centering,margin={5cm,-2cm}}
% \usepackage[utf8]{inputenc} is no longer required (since 2018)

%Set the font (output) encoding
%--------------------------------------
\usepackage[T1]{fontenc} %Not needed by LuaLaTeX or XeLaTeX
%--------------------------------------

%German-specific commands
%--------------------------------------
\usepackage[ngerman]{babel}

%Hyphenation rules
%--------------------------------------
\usepackage{hyphenat}
\hyphenation{Mathe-matik wieder-gewinnen}
%--------------------------------------

\setlength{\parindent}{0pt}
\setlength{\parskip}{0.8em}

\title{Mathe Merkheft
{\\\Large Herzog-Christoph-Gymnasium Beilstein}}
\author{Luke Moll}
\date{Abitur Schuljahr 2025}

\begin{document}
\newtheorem{satz}{Satz}
\newtheorem{definition}{Definition}
\maketitle
\begin{abstract}
    Dieses Dokument fasst alle wichtigen Dinge aus dem Mathe-LK-Unterricht zusammen, gegliedert in zwei Teile: \\ 1. Merkheft mit Definitionen, Sätzen und Erklärungen. Die Beispiele in diesem Teil dienen meist nur zur Erklärung und gehen nicht in die oft gebrauchte Tiefe. \\ 2. Hier findet man weiter Verweise zu wichtigen Aufgaben, welche sich im Schulbuch \textit{Lambacher Schweizer Mathematik Kursstufe - Leistungsfach} finden.
\end{abstract}
\tableofcontents
\vspace{2cm} %Add a 2cm space

\chapter{Grundlagen der Differenzialrechnung}
\section{Ableitung und Ableitungsregeln}

Gegeben ist $f$ auf $[a;b]$ 

\begin{tikzpicture}
\begin{axis}[clip=false, 
    xmin=0, xmax=4, ymin=0, ymax=10,
    axis lines = middle, 
    xlabel=$x$, ylabel=$y$,
    xtick={0,1,2,3,4}, xticklabels={0,a, ,b},
    ytick={0,2,4,6,8,10}, yticklabels={0,$f(a)$, $f(b)$}, grid = major]
\addplot[color=black, samples=50, domain=0.5:4]{0.5*(x-1)^2+2}node[right, pos = 0.8]{$G_f$};
\addplot[color=red, domain= 0.5:3.5]{x+1};
\draw [red, fill=yellow]  (1,2) node[thick, cross, red] {} node[below] {$P$};
\draw [red, fill=yellow]  (3,4) node[thick, cross, red] {} node[below] {$Q$};
\end{axis}
\end{tikzpicture}

$$\frac{f(b)-f(a)}{b-a}$$
heißt Differenzenqoutint von $f$ in $[a;b]$. Der Differenzenqoutient ist die Steigung m der Geraden durch $P(a|f(a))$ und $Q(b|f(b))$\footnotemark{}.

\flushbottom\footnotetext{genannt Sekantensteigung oder auch mittlere Änderungsrate}

\begin{tikzpicture}
\begin{axis}[clip=false, 
    xmin=0, xmax=4, ymin=0, ymax=10,
    axis lines = middle, 
    xlabel=$x$, ylabel=$y$,
    xtick={0,1,2,3,4}, xticklabels={0,$x_0$, ,$x_0+h$},
    ytick={0,2,4,6,8,10}, yticklabels={0,$f(x_0)$, $f(x_0+h)$}, grid = major]
\addplot[color=black, samples=50, domain=0.5:4]{0.5*(x-1)^2+2}node[right, pos = 0.8]{$G_f$};
\addplot[color=red, domain= 0.5:3.5]{x+1};
\draw [red, fill=yellow]  (1,2) node[thick, cross, red] {} node[below] {$P$};
\draw [red, fill=yellow]  (3,4) node[thick, cross, red] {} node[below] {$Q$};
\end{axis}
\end{tikzpicture}

Falls sich der Differenzenquotient $\frac{f(x_{0}+h)-f(x_{0})}{h}$ für $h \to 0$ einem festen Wert nähert, dann heißt dieser sogenannte Grenzwert Ableitung von $f$ an der Stelle $x_0$. $f$ ist an der Stelle $x_0$ differenzierbar. 

Geometrich: Tangentensteigung im Punkt $P(x_0|f(x_0))$.

Die Ableitung $f^{\prime}(x)$ ist also definiert durch:
$$f^{\prime}(x)=\lim_{h\to 0} \frac{f(x_{0}+h)-f(x_{0})}{h}$$
Eine Funktion, die jeder Stelle x die Ableitung $f^{\prime}(x)$ an dieser stelle zuordnet, heißt Ableitungsfunktion $f^{\prime}(x)$.

\noindent\rule{\textwidth}{1pt}

\textit{Beispiele:}

(1) S.12 Nr. 1 b)
\begin{equation*}
\begin{gathered}
    f(t)  = -4\sqrt{t}-3t \\
    = -4t^{\frac{1}{2}}-3t \\
    f^{\prime}(t)  = -2t^{-\frac{1}{2}}-3 \\
    f^{\prime\prime}(t)  = t^{-\frac{3}{2}} = \frac{1}{\sqrt{t^3}}
\end{gathered}
\end{equation*}
\begin{align*}
    D_f & = \mathbb{R}_0^+ = [0;\infty[ \ = [0;\infty) \text{\footnotemark}\\
    D_{f^\prime} & = \mathbb{R}^+ = \ ]0;\infty[ \\
    D_{f^{\prime\prime}} & = \mathbb{R}^+
\end{align*}
\footnotetext{alte Schreibweise}

e)
\begin{equation*}
    \begin{gathered}
        f(x)  = -\cos(x) + \frac{1}{x^2} = -\cos(x)+x^{-2} \\
    f^{\prime}(x)  = \sin(x) -2x^{-3} \\
    f^{\prime\prime}(x)  = \cos(x)+6x^{-4}
    \end{gathered}
\end{equation*}
\begin{align*}
    D_f = \mathbb{R} \ \backslash \ \{0\} = D_{f^{\prime}} = D_{f^{\prime\prime}}
\end{align*}

(2) S. 12 Nr. 2 b)
\begin{equation*}
    \begin{gathered}
        f(t)  = \frac{6t^3-t^2}{2t} = \frac{2t(3t^2 - \frac{1}{2}t)}{2t} = 3t^2 - \frac{1}{2}t \\
        f^{\prime}(x)  = 6t - \frac{1}{2}
    \end{gathered}
\end{equation*}
\begin{align*}
    D_f & = \mathbb{R} \ \backslash \ \{0\}  \\
    D_{f^{\prime}} & = \mathbb{R}
\end{align*}

e)
\begin{equation*}
    \begin{gathered}
        f(s)  = (2s-4)\cdot(4-s) = 8s-2s^2-16+4s = -2s^2 + 8s -16\\
        f^{\prime}(s)  = -4s +8 
    \end{gathered}
\end{equation*}
\begin{align*}
    D_f = \mathbb{R} = D_{f^{\prime}}
\end{align*}

\section{Verketten von Funktionen}
Vorüberlegung:

$u(x)= x^2 \ , \ v(x) = x + 2$

Summen von $u$ und $v$:
$$u(x)+v(x) = x^2 +x+2 = (u + v)(x)$$

Differenz von $u$ und $v$:
$$u(x)-v(x)=x^2-x-2 = (u-v)(x)$$

Produkt von $u$ und $v$:
$$u(x) \cdot v(x) = x^3+ 2x = (u \cdot v)(x)$$

Quotient von $u$ und $v$:
$$\frac{u(x)}{v(x)} = \frac{x^2}{x+2}$$

\begin{definition}
Unter der Verkettung der beiden Funktionen $u$ und $v$ versteht man die Funktion $f$ mit
$$f(x)=u(v(x))$$
\end{definition}

$u$ heißt äußere Funktion, $v$ heißt innere Funktion der Verkettung. Schreibweise: $f = u \circ v$ (lies: \glqq $u$ nach $v$\grqq{} oder  \glqq $u$ verkettet $v$\grqq{} ).

\noindent\rule{\textwidth}{1pt}

\textit{Beispiele:}

$u(x) = x^2 \ , \ v(x) = x + 2$

$f(x) = u(v(x)) = (x+2)^2 \\ g(x) = v(u(x)) = x^2 + 2$

\textbf{Beachte:} Die Verkettung zweier Funktionen ist i. a. nicht kommutativ d. h. $u \circ v \neq v \circ u$.

\section{Kettenregel}

\begin{satz}
    Ist $f = u \circ v$ eine Verkettung zweier differenzirbarer Funktionen $u$ und $v$ so ist $f$ auch differenzierbar und es gilt:

$$f^{\prime}(x) = u^{\prime}(v(x)) \cdot v^{\prime}(x)$$
\end{satz}

kurz: \glqq Ableitung der Gesamtfunktion ist gleich äußere Ableitung mal innere Abeitung.\grqq{}

\noindent\rule{\textwidth}{1pt}

\textit{Beispiele:}

(1)
\begin{equation*}
    \begin{gathered}
        f(x)  = (5-3x)^4 \\
        f^{\prime}(x)  = 4(5-3x)^3 \cdot (-3) = -12(5-3x)^3
    \end{gathered}
\end{equation*}

(2)
\begin{equation*}
    \begin{gathered}
        f(x)  = \sqrt{x^2 + 1} = (x^2 + 1)^\frac{1}{2} \\
        f^{\prime}(x)  = \frac{2x}{2\sqrt{x^2 + 1}} = \frac{x}{\sqrt{x^2 + 1}}
    \end{gathered}
\end{equation*}

(3)
\begin{equation*}
    \begin{gathered}
        f(x)  = \frac{3}{2x^2 - 1} = 3(2x^2 - 1)^{-1} \\
        f^{\prime}(x)  = \frac{-12x}{(2x^2 - 1)^2}
    \end{gathered}
\end{equation*}

(4)
\begin{equation*}
    \begin{gathered}
        f(x)  = 3\sin(2x^3) \\
        f^{\prime}(x)  = 18x^2\cos(2x^3)
    \end{gathered}
\end{equation*}

\section{Produktregel}

\begin{satz}
Sind die Funktionen $u$ und $v$ differenzierbar, so ist auch die Funktion $f$ mit $f = u \cdot v$ differenzierbar, und es gilt \footnote{sehr selten: $f(x) = u\cdot v\cdot w \quad , \quad f^\prime(x) = u^\prime \cdot v \cdot w + u \cdot v^\prime \cdot w + u \cdot v \cdot w^\prime$}:
$$f^\prime(x) = u^\prime(x) \cdot v(x) + u(x) \cdot v^\prime(x)$$
\end{satz}

\noindent\rule{\textwidth}{1pt}

\textit{Beispiele:}

(1) 
\begin{equation*}
    \begin{gathered}
    f(x) = x^2 \sin(x) \\
    f^\prime(x) = 2x \sin(x) + x^2 \cos(x)
    \end{gathered}
\end{equation*}

(2) 
\begin{equation*}
    \begin{gathered}
     f(x)  = \sqrt{x}(4x + 1) \\
     f^\prime(x)  = \frac{4x + 1}{2\sqrt{x}} + 4 \sqrt{x}    
    \end{gathered}
\end{equation*}

(3) 
\begin{equation*}
    \begin{gathered}
     f(x)  = (2x - 3) \cos(4x) \\
     f^\prime(x)  = 2\cos(4x)-4(2x-3)\sin(4x)    
    \end{gathered}
\end{equation*}

\ \\
\ \\
\ \\
\ \\
\ \\

(4) Der Graph von $f$ hat im Punkt $P(0|1)$ eine waagrechte Tangente. Untersuchen Sie, ob dies auch für den Graph der Funktion $g$ mit $g(x) = f(x) \cdot \cos(x)$ gilt.\\

$G_f: P(0|1) \in G_f$ mit waagrechter Tangente $\Rightarrow f(0) = 1$ und $f^\prime(0) = 0$ \\
zu zeigen: $g(0) = 1$ und $g^\prime(0) = 0$
\begin{equation*}
    \begin{gathered}
        g(x) = f(x) \cdot \cos(x) \\
        g^\prime(x) = f^\prime(x) \cdot \cos(x) - f(x) \cdot \sin(x)
    \end{gathered}
\end{equation*} 
Also:
\begin{equation*}
    \begin{gathered}
        g(0) = f(0) \cdot \cos(0) = 1 \cdot 1 = 1 \Rightarrow P(0|1) \in G_g \tag{1} \label{1} \\
    \end{gathered}
\end{equation*}
\begin{equation*}
    \begin{gathered}
        g^\prime(0) = f^\prime(0) \cdot \cos(0) - f(0) \cdot \sin(0) = 0 \Rightarrow \text{$G_g$ hat bei $P$ eine waagrechte Tangente} \tag{2} \label{2}
    \end{gathered}
\end{equation*}

Aus \eqref{1} und \eqref{2} folgt die Behauptung.

\section{Monotonie und Krümmung}
Wiederholung:

Gegeben ist der Graph $G_f$:

\begin{tikzpicture}
\draw[step=1.0,lightgray,very thin] (0,0) grid (7.2,5.8);
\begin{axis}[clip=true, 
    xmin=0, xmax=8, ymin=0, ymax=10,
    axis lines = middle, 
    xlabel=$x$, ylabel=$y$,
    xtick={3.92,6.06, 6.76}, xticklabels={$x_1$, $x_2$, $x_3$},
    ytick={6.22, 7.26, 8.61}, yticklabels={$f(x_1)$,$f(x_3)$, $f(x_2)$}]
\addplot[color=black, samples=100, domain=0.7:7.5]{-0.05*(x-2)^2*3*(0.8*(x-6)^2-0.25)+8}node[right, pos = 0.8]{$G_f$};
\draw [red, fill=yellow]  (3.92,6.22) node[thick, cross, red] {} node[below] {$x_1$};
\draw [red, fill=yellow]  (6.08,8.61) node[thick, cross, red] {} node[below] {$x_2$};
\draw [red, fill=yellow]  (6.76, 7.26) node[thick, cross, red] {} node[right] {$x_3$};
\end{axis}
\end{tikzpicture}
\begin{definition}
    Wenn für alle $x_1$, $x_2$ mit $x_1 < x_2$ auf einem Intervall $I$ gilt:
    
    $f(x_1) < f(x_2)$, dann heißt $f$ streng monton wachsend auf I

    $f(x_1) > f(x_2)$, dann heißt $f$ streng monton fallend auf I \footnote{Beachte: die Ränder von Monotonierintervallen kann man zu beiden Monotonieintervallen dazunehmen. Intervall, in dem die Funktion $f$ \\ (1) streng monoton wachsend ist: $[x_1;x_2]$ \\(2) strend monoton fallend ist: $[x_2;x_3]$} \\
\end{definition}
\begin{satz}[Monotoniesatz]
    Die Funktion $f$ sei im Intervall $I$ differenzierbar. Wenn für alle $x \in I$ gilt:

    $f^\prime(x) > 0$, dann ist die Funktion $f$ streng monoton wachsend auf $I$. 
    
    $f^\prime(x) < 0$, dann ist die Funktion $f$ streng monoton fallend auf $I$. \footnote{Die Umkehrung des Monotoniesatzes gilt nicht.} \\
\end{satz}

\begin{figure}[h]
    \centering
    \captionsetup[subfigure]{oneside,margin={-1cm,0cm}}
    \begin{subfigure}[h]{0.49\textwidth}
        \begin{tikzpicture}
        \begin{axis}[clip=false, 
            xmin=0, xmax=6, ymin=0, ymax=10,
            axis lines = middle, 
            xlabel=$x$, ylabel=$y$,
            xtick={}, xticklabels={},
            ytick={}, yticklabels={},
            width=\linewidth, grid = major]
        \addplot[color=black, samples=50, domain=0.5:5.5]{-(x-3)^2 +7.5};
        \end{axis}
        \end{tikzpicture} 
    \caption*{Rechtskurve}
    \end{subfigure}
    \begin{subfigure}[h]{0.49\textwidth}
        \begin{tikzpicture}
        \begin{axis}[clip=false, 
            xmin=0, xmax=6, ymin=0, ymax=10,
            axis lines = middle, 
            xlabel=$x$, ylabel=$y$,
            xtick={}, xticklabels={},
            ytick={}, yticklabels={},
            width=\linewidth, grid = major]
        \addplot[color=black, samples=50, domain=0.5:5.5]{(x-3)^2 +1};
        \end{axis}
        \end{tikzpicture} 
    \caption*{Linkskurve}
    \end{subfigure}
    \label{fig:enter-label}
\end{figure}

\begin{tabularx}{\textwidth} { 
  | >{\raggedright\arraybackslash}X 
  | >{\centering\arraybackslash}X 
  | >{\centering\arraybackslash}X | }
 \hline
  & Rechtskurve & Linkskurve \\
 \hline
 $f^\prime(x)$ ist streng monoton  & fallend & steigend  \\
 \hline
 $f^{\prime\prime}(x)$ & < 0 & > 0 \\
\hline
\end{tabularx}

\noindent\rule{\textwidth}{1pt}

\textit{Beispiel: Bestimmung des Krümmungsverhaltens}

\begin{equation*}
    \begin{gathered}
        f(x) = \frac{1}{4}x^4-\frac{4}{3}x^3+2x^2+2 \\
        f^\prime(x) = x^3-4x^2+4x \\
        f^{\prime\prime}(x) = 3x^2-8x + 4
    \end{gathered}
\end{equation*}

\begin{equation*}
    \begin{gathered}
        f^{\prime\prime}(x) = 0 \\
        0 = 3x^2-8x + 4
    \end{gathered}
\end{equation*}

Mitternachtsformel:

\begin{equation*}
    \begin{gathered}
        x_{1,2}=\frac{8\pm\sqrt{(-8)^2-4\cdot3\cdot4}}{2\cdot3} \Rightarrow x_1 = 2 \ , \ x_2 = \frac{2}{3}
    \end{gathered}
\end{equation*}

Das heißt: 

\begin{tabularx}{\textwidth} { 
  | >{\raggedright\arraybackslash}X 
  | >{\centering\arraybackslash}X
  | >{\centering\arraybackslash}X
  | >{\centering\arraybackslash}X
  | >{\centering\arraybackslash}X
  | >{\centering\arraybackslash}X | }
 \hline
  & $x < \frac{2}{3}$ & $x = \frac{2}{3}$ & $\frac{2}{3} < x < 2$ & $x = 2$ & $x > 2$ \\
 \hline
 $f^{\prime\prime}(x)$  & $>0$ & $0$ & $<0$ & $0$ & $>0$  \\
 \hline
 $f^{\prime}(x)$ & streng monoton wachsend & - & streng monoton fallend & - & streng monoton wachsend \\
 \hline
 $G_f$ & Linkskurve & - & Rechtskurve & - & Linkskurve \\
\hline
\end{tabularx}

\section{Extrem- und Wendepunkte}
\textbf{Flussdiagramm:} Bestimmung der lokalen Extrema einer Funktion $f$

Die Funktion $f$ sei zweimal differenzierbar.\\

\begin{tikzpicture}[node distance=2cm, text width= 4cm]
    \node (pro1) [process, text width=4cm] {notwendige Bedingung $f^\prime(x) = 0$};
    \node (pro2a) [process, text width=4cm, below of=pro1] {es existiert \\ eine Lösung};
    \node (pro2b) [process, text width=4cm, below of=pro1, xshift = 8cm] {es existiert keine Lösung};
    \node (pro3a) [process, text width=4cm, below of=pro2a] {hinreichende Bedingung $f^\prime(x) = 0 \land f^{\prime\prime}(x) \neq 0$};
    \node (pro3b) [process, text width=4cm, below of=pro2b] {keine Extrema};
    \node (pro4a) [process, text width=4cm, below of=pro3a, xshift = 1cm, yshift= -0.5cm] {$f^{\prime\prime}(x) < 0$ \\ x ist Maximumstelle};
    \node (pro4b) [process, text width=4cm, below of=pro3a, xshift = -4cm, yshift= -0.5cm] {$f^{\prime\prime}(x) > 0$ \\ x ist Minimumstelle};
    \node (pro4c) [process, text width=4cm, below of=pro3a, xshift= 7cm,  yshift= -0.5cm] {$f^{\prime\prime}(x) = 0$};
    \node (pro5) [process, text width=4cm, below of=pro4c] {VZW-Kriterium\footnotemark \\ von $f^\prime$ in der Umgebung von $x$};
    \node (pro6a) [process, text width=4cm, below of=pro5, xshift= -2cm, yshift= -0.5cm] {VZW von \\
                                                        $-$ nach $+$ \\
                                                        $x$ ist Minimumstelle};
    \node (pro6b) [process, text width=4cm, left of=pro6a, xshift = -2.5cm] {VZW von \\
                                                        $+$ nach $-$ \\
                                                        $x$ ist Maximumstelle};
    \node (pro6c) [process, text width=4cm, right of=pro6a, xshift = 2.5cm] {kein VZW \\
                                                        \quad \newline
                                                        keine Extrema};
    \node (pro7) [process, text width=4cm, below of=pro4b, yshift = -4.5cm] {Berechne $f(x)$};

    \draw [arrow] (pro1) -- (pro2a);
    \draw [arrow] (pro1) -| (pro2b);
    \draw [arrow] (pro2a) -- (pro3a);
    \draw [arrow] (pro2b) -- (pro3b);
    \draw [arrow] (pro3a) -- (pro4a);
    \draw [arrow] (pro3a) -- (pro4b);
    \draw [arrow] (pro3a) -- (pro4c);
    \draw [arrow] (pro4c) -- (pro5);
    \draw [arrow] (pro5) -- (pro6a);
    \draw [arrow] (pro5) -- (pro6b);
    \draw [arrow] (pro5) -- (pro6c);
    \draw [arrow] (pro4a) -- (pro7);
    \draw [arrow] (pro4b) -- (pro7);
    \draw [arrow] (pro6a) |- (pro7);
    \draw [arrow] (pro6b) -- (pro7);
\end{tikzpicture}

\footnotetext{VZW = Vorzeichenwechsel}

\textit{Beispiel:}

\begin{equation*}
    \begin{gathered}
        f(x) = -\frac{1}{4}x^4 + \frac{1}{3}x^3 + 1
    \end{gathered}
\end{equation*}

\textbf{Extremepunkte:}

\begin{equation*}
    \begin{gathered}
        f^\prime(x)=-x^3+x^2 \\
        f^{\prime\prime}(x)=-3x^2 + 2x
    \end{gathered}
\end{equation*}

notwendige Bedingung: $f^\prime(x) = 0$

\begin{equation*}
    \begin{gathered}
        0 = -x^3+x^2 \\
        0 = x^2  (-x+1) \\
        \Rightarrow \text{Satz vom Nullprodukt: } x_1 = 0, \ x_2 = 1
    \end{gathered}
\end{equation*}

hinreichende Bedingung: $f^\prime(x) = 0 \land f^{\prime\prime}(x)\neq0$

\begin{equation*}
    \begin{gathered}
        f^{\prime\prime}(0)=0 \Rightarrow \text{keine Aussage möglich}
    \end{gathered}
\end{equation*}

VZW von $f^\prime$ in der Umgebung von $x_1 = 0$:
\begin{equation*}
    \begin{gathered}
        \text{für } x < 0 \text{ gilt: } f^\prime(x)=x^2(-x+1) > 0 \\
        \text{für } x > 0 \text{ gilt: } f^\prime(x)=x^2(-x+1) > 0 \\
        \Rightarrow f \text{ hat keinen VZW an der Stelle } x_1 = 0 \text{ , also ist } x_1 \text{ auch keine Extremstelle.}
    \end{gathered}
\end{equation*}

\begin{equation*}
    \begin{gathered}
        f^{\prime\prime}(1) = -1 \neq 0 \\
        f^{\prime\prime}(1) < 0 \Rightarrow \text{Maximumstelle}
    \end{gathered}
\end{equation*}

\begin{equation*}
    \begin{gathered}
        f(1) = \frac{13}{12} \ \Rightarrow \text{Hochpunkt } H\left(1 \ \middle| \ \frac{13}{12}\right)
    \end{gathered}
\end{equation*}

\ \\
\ \\
\ \\
\ \\
\ \\
\ \\
\ \\

\textbf{Wendepunkte:}

\begin{equation*}
    \begin{gathered}
        f^{\prime\prime}(x)=-3x^2 + 2x \\
        f^{\prime\prime\prime}(x) = -6x + 2
    \end{gathered}
\end{equation*}

notwendige Bedingung: $f^{\prime\prime}(x)=0$

\begin{equation*}
    \begin{gathered}
        0 = -3x^2 + 2x \\
        0 = -x(3x-2) \\
        \Rightarrow \text{Satz vom Nullprodukt: } x_1 = 0, \ x_2 = \frac{2}{3}
    \end{gathered}
\end{equation*}

hinreichende Bedingung: $f^{\prime\prime}(x)=0 \land f^{\prime\prime\prime}(x) \neq 0$

\begin{equation*}
    \begin{gathered}
        f^{\prime\prime\prime}(0)= 2 \neq 0 \Rightarrow \text{Wendestelle} \\
        f^{\prime\prime\prime}\left(\frac{2}{3}\right)= -2 \neq 0 \Rightarrow \text{Wendestelle}
    \end{gathered}
\end{equation*}

Damit:

\begin{equation*}
    \begin{gathered}
        W_1(0 \ | \ f(0)) \text{, also: } W_1(0 \ | \ 1) \\
        W_2\left(\frac{2}{3} \ \middle| \ f\left(\frac{2}{3}\right)\right) \text{, also: } W_2\left(\frac{2}{3} \ \middle| \ \frac{85}{81}\right) \\
    \end{gathered}
\end{equation*}

\textbf{Anmerkung:}

Bei $W_1$ handelt es sich um einen Sattelpunkt\footnote{= Wendepunkt mit waagrechter Tangente}.


\section{Tangente und Normale}

\begin{definition}
    Die Gerade, die senkrecht zu einer Tangente verläuft heißt Normale. Sie hat die Steigung 
    $$-\frac{1}{m_t} = -\frac{1}{f^\prime(x)} \qquad \text{, falls } f^\prime(x) \neq 0 \text{ ist.}$$
\end{definition}

\pagebreak

\textit{Beispiel:}

Gesucht: Gleichung der Normalen an $G_f$ im Punkt $P\left(2\middle|\frac{1}{2}\right)$.

\begin{equation*}
    \begin{gathered}
        f(x) = \frac{1}{x} \\
        f^\prime(x) = -\frac{1}{x^2} \\
        f^\prime(2) = -\frac{1}{2^2} = -\frac{1}{4} = m_t
    \end{gathered}
\end{equation*}

Also:

\begin{equation*}
    \begin{gathered}
        m_n = 4 \Rightarrow n(x) = 4x + c \\
        P \text{ in } n(x): \ \frac{1}{2} = 4\cdot2 + c \Leftrightarrow c = -\frac{15}{2} \\
        n(x) = 4x - \frac{15}{2}
    \end{gathered}
\end{equation*}

\section{Extremwertprobleme mit Nebenbedingung}
\subsection{Extremwertproblem: Volumenminimierung, Oberflächenminimierung, etc.}

\textit{Beispiel: Getränkedose}

Eine handelsübliche (zylindrische) Cola-Dose enthält 0,33 l des Getränks, insgesamt beträgt das Volumen 0,35 l. Ein Hersteller möchte aus Umweltschutz- und Kostengründen seine Dose so konstuieren, dass er so wenig Verpackungsmaterial (Aluminium) wie möglich benötigt.

Bestimme wie groß dann der Radius und die Höhe der \glqq optimalen\grqq{} Dose sein müssen.

Ziel: Minimierung der Dosenoberfläche 

\begin{equation*}
    A = 2\pi \cdot r^2 + U \cdot h \tag{1}\label{3}
\end{equation*}
\begin{equation*}
    V = G \cdot h = \pi r^2 \cdot h = 0,35 \tag{2}\label{4}
\end{equation*}

Auflösen der Nebenbedingung \eqref{4} nach $h$:

\begin{equation*}
    h = \frac{0,35}{\pi r^2} \label{5}\tag{3}
\end{equation*}

\eqref{5} in \eqref{3} einsetzen: $$A(r) = 2\pi r^2 + 2\pi r\cdot\frac{0,35}{\pi r^2} = 2\pi r^2 + \frac{0,7}{r} \qquad \qquad \text{ ,mit } D_A = \ ]0;\infty[ \ = \mathbb{R}^+ $$

\begin{equation*}
    \begin{gathered}
        A(r) = 2\pi r^2 + 0,7r^{-1} \\
        A^\prime(r) = 4\pi r - 0,7^{-2} \\
        A^{\prime\prime}(r) = 4\pi + 1,4r^{-3}
    \end{gathered}
\end{equation*}

notwendige Bedingung: $A^\prime(r) = 0$

\begin{equation*}
    \begin{gathered}
        0 = 4\pi r - 0,7^{-2} \\
        0 = 4\pi r^3 - 0,7 \\
        0,7 = 4\pi r^3 \\
        r = \sqrt[3]{\frac{0,7}{4\pi}}
    \end{gathered}
\end{equation*}

hinreichende Bedingung: $A^\prime(r) = 0 \land A^{\prime\prime}(r) \neq 0$

\begin{equation*}
    \begin{gathered}
        A^{\prime\prime}\left(\sqrt[3]{\frac{0,7}{4\pi}}\right) \approx 12\pi > 0 \Rightarrow \text{ lokale Minimumstelle} \\
        A\left(\sqrt[3]{\frac{0,7}{4\pi}}\right) \approx 2,749
    \end{gathered}
\end{equation*}

Randwertbetrachtung: 

$\text{für }  r \to 0  \text{ gilt: }  \underbrace{2\pi r^2}_{\to 0} + \underbrace{\frac{0,7}{r}}_{\to \infty} \to \infty \\
\text{für }  r \to \infty \text{ gilt: }  \underbrace{2\pi r^2}_{\to \infty} + \underbrace{\frac{0,7}{r}}_{\to 0} \to \infty$

$\Rightarrow$ keine weiteren Minima

Also: 

$r = \sqrt[3]{\frac{0,7}{4\pi}}$ in \eqref{5}: $ h \approx 0, 76 $

Die \glqq optimale\grqq{} Dose hat folglich eine Radius von ca. 0,38 cm und eine Höhe von ca. 0,76 cm.

\subsection{Extremwertproblem: Kleinster Abstand eines Punktes zu einem Graphen}

Hier soll der kleinste Abstand eines Punktes $P(a \ |  \ b)$ von dem Graph einer Funktion $f$ bestimmt werden. Dazu wird auf dem Graph ein belibiger Punkt $Q(u \ | \ f(u))$ ausgewählt und mit dem \textbf{Satz des Pythagoras} der Abstand zu $P(P_x \ |  \ P_y)$ berechnet.\\
Dieser Abstand ist die Zielfunktion $z(u)$ deren kleinster Wert zu bestimmen ist:
$$z(u) = \sqrt{(u-P_x)^2 + (f(u)-P_y)^2}$$
Die \textbf{Nebenbedingungen} sind hier die Seitenlängen $u - P_x$ und $f(u) - P_y$ des dargestellten Dreiecks.

\begin{tikzpicture}
\begin{axis}[clip=false, 
    xmin=-3, xmax=8.5, ymin=-2, ymax=5.5,
    axis lines = middle, 
    xlabel=$x$, ylabel=$y$,
    xtick={-2,-1,0,1,2,3,4,4.5,5,6,7,8}, xticklabels={-2,-1,0,1,$P_x$,3,4,$u$,5,6,7,8},
    ytick={-1,0,1,2,3,3.25,4,5}, yticklabels={-1,0,1,$P_y$, ,$f(u)$,4,5},
    width=16.5cm, height = 12cm]
    \addplot[thick, color=black, samples=50, domain=3.5:6.5]{(x-5)^2+3}node[right, pos = 0.9]{$G_f$};
    \draw (4.60,3.25) node[anchor= south west, fill=white] {$Q(u  | f(u))$};
    \draw (3.25,2.00) node[below] {$a=u - P_x$};
    \draw (2.60,2.35) node[above] {$z(u)$};
    \draw (4.50,2.50) node[right] {$b=f(u)-P_y$};
    \draw [dashed] (2,2) -- (2,0);
    \draw [dashed] (2,2) -- (0,2);
    \draw [dashed] (4.5,2) -- (4.5,0);
    \draw [dashed] (4.5,3.25) -- (0,3.25);

    \addplot[thick, mark=,black] coordinates {(4.5,3.25) (2,2) (4.5,2) (4.5,3.25)};
    
    \draw (4.50,3.25) node[thick, cross,red] {};
    \draw (2,2) node[thick, cross,red] {} node[anchor= north east] {$P(P_x | P_y)$};
    \draw (4.5,2) node[thick, cross,red] {};
\end{axis}

\end{tikzpicture}


\noindent\rule{\textwidth}{1pt}

\textit{Beispiel:}\\
Betrachtet werden die Funktion $f$ mit $f(x) = 0,5x^2 + 1$ und der Punkt $P(4 \ | \ 2)$. Bestimme den Punkt $Q$ auf dem Grapfen von $f$, der den kleinsten Abstand von $P$ hat. Gib den kleinsten Abstand an.

\pagebreak

(1) Ggf. eine Skizze erstellen nach dem oben gegebenen Schema. \\

\begin{tikzpicture}
\begin{axis}[clip=false, 
    xmin=-3.5, xmax=6.5, ymin=-0.5, ymax=4.5,
    axis lines = middle, 
    xlabel=$x$, ylabel=$y$,
    xtick={-3, -2,-1,0,1,2,3,4,5,6}, xticklabels={-3, -2,-1,0,1,$u$,3,4,5,6},
    ytick={0,1,2,3,4}, yticklabels={0,1,2,$f(u)$,4},
    width=11cm, height = 8cm]
    \addplot[thick, color=black, samples=50, domain=-2.5:2.5]{0.5*x^2 + 1}node[right, pos = 0.95]{$G_f$};

    \addplot[thick, mark=,black] coordinates {(2,3) (4,2) (2,2) (2,3)};

    \draw (2,3) node[anchor= south east] {$Q$};
    \draw (2,2.5) node[right] {$b$};
    \draw (3,2) node[below] {$a$};
    \draw (3.3,2.4) node[right] {$z$};

    \draw [dashed] (2,2) -- (2,0);
    \draw [dashed] (2,3) -- (0,3);
    
    \draw (2,3) node[thick, cross, red] {};
    \draw (2,2) node[thick, cross, red] {};
    \draw (4,2) node[thick, cross, red] {} node[right] {$P$};

\end{axis}
\end{tikzpicture}

(2) Beschreibe den formalen Zusammenhang der betrachteten Größen. Nutze dazu die Bezeichnungen aus deiner Skizze.\\
$z^2 = b^2 + a^2$ \newline

(3) Schreibe die Nebenbedingung auf. \\
$a = P_x -u \\ b = f(u) - P_y$ \newline

(4) Setze die Nebenbedingung ein und stelle die Zielfunktion auf. \\
$z^2 = (4-u)^2 + ((0,5u^2+1)-2)^2$ \\
Also: $$z(u) = \sqrt{(4-u)^2 + ((0,5u^2-1)^2} = \sqrt{16-8u+u^2+0,25u^4-u^2+1} $$
$$z(u) = \sqrt{0,25u^4 -8u +17}$$

(5) Leite die Zielfunktion ab und bestimme mögliche Extremstellen. \\
$$z^\prime(u) = \frac{u^3 -8}{2\sqrt{0,25u^4 -8u +17}}$$

notwendige Bedingung: $z^\prime(u) = 0$
\begin{equation*}
    \begin{aligned}
        0 & = \frac{u^3 -8}{2\sqrt{0,25u^4 -8u +17}} & \qquad & |\cdot 2\sqrt{0,25u^4 -8u +17} \\
        0 & = u^3 - 8 & \qquad & | +8 \\
        8 & = u^3 & & | \sqrt[3]{} \\
        u & = 2
    \end{aligned}
\end{equation*}

(6) Bestimme die Art der Extremstelle. \\

VZW-Kriterium: \\
für $u>2 \land u \to 2$ gilt: $z(u) < 0$ \\
für $u<2 \land u \to 2$ gilt: $z(u) > 0$ \\
$\Rightarrow$ VZW von $-$ nach $+ \Rightarrow u = 2$ ist Minimumstelle

(7) Bestimme die Koordinaten von Q.\\

Da $Q \in G_f$: $f(2) = 0,5\cdot2^2 +1 = 3$\\
Also: $Q(2 \ | \ 3)$

(8) Bestimme den kleinsten Abstand.\\

$z(2) = \sqrt{5} \approx 2,24$
\chapter{Lineare Gleichungen}
\section{Der Gauß-Algorithmus}

Ziel: LGS\footnote{= lineares Gleichungssystem, meistens drei Gleichungen mit drei Unbekannten} mithilfe von Äquivalenzumformungen auf Stufen-/Dreiecksform bringen\\

\begin{minipage}{0.6\textwidth}
    $\begin{gmatrix}
    3x_1 & + & 6x_2 & - & 2x_3 & = & -4 \\
    3x_1 & + & 2x_2 & + & x_3 & = & 0 \\
    \frac{3}{2}x_1 & + & 5x_2 & - & 5x_3 & = & -9 \\
    \rowops
    \mult{0}{(-1)}\add{0}{1}
    \mult{2}{(-2)}\add{0}{2}
    \end{gmatrix}$
    
    \par\noindent\rule{0.65\textwidth}{0.4pt}
    
    $\begin{gmatrix}
    3x_1 & + & 6x_2 & - & 2x_3 & = & -4 \\
    & - & 4x_2 & + & 3x_3 & = & 4 \\
    & - & 4x_2 & + & 8x_3 & = & 14 \\
    \rowops
    \mult{1}{(-1)}\add{1}{2}
    \end{gmatrix}$
    
    \par\noindent\rule{0.65\textwidth}{0.4pt}
    
    $\begin{gmatrix}
    3x_1 & + & 6x_2 & - & 2x_3 & = & -4 \\
    & - & 4x_2 & + & 3x_3 & = & 4 \\
    & & & & 5x_3 & = & 10\\
    \end{gmatrix}$
    
    \par\noindent\rule{0.65\textwidth}{0.4pt}
    
    $\begin{gmatrix}
    \text{Also:} & \quad & \ & \ & \ & x_3 & = & 3 \\
    & \quad & \ & \ & \ & x_2 & = & 0,5 \\
    & \quad & \ & \ & \ & x_1 & = & -1 
\end{gmatrix}$

\ \newline
Damit: $\mathbb{L}$ = \{$(-1; \ 0,5; \ 2)$\}

\end{minipage}
\begin{minipage}{0.4\textwidth}
    \textbf{Matrixform\footnotemark:}

\begin{equation*}
    \begin{aligned}
        \begin{gmatrix}[p]
        3 \ & 6 \ & -2 & \BAR & -4 \\
        3 \ & 2 \ & 1 & \BAR & 0 \\
        \frac{3}{2} \ & 5 \ & -5 & \BAR & -9 
        \rowops
        \mult{0}{(-1)}\add{0}{1}
        \mult{2}{(-2)}\add{0}{2}
        \end{gmatrix}
        
        \par\noindent\rule{0.7\textwidth}{0.4pt}
        
        \begin{gmatrix}[p]
        3 & 6 & -2 & \BAR & -4 \\
        0 & -4 & 3 & \BAR & 4 \\
        0 & -4 & 8 & \BAR & 14 
        \rowops
        \mult{1}{(-1)}\add{1}{2}
        \end{gmatrix}
        
        \par\noindent\rule{0.7\textwidth}{0.4pt}
        
        \begin{gmatrix}[p]
        3 & 6 & -2 & \BAR & -4 \\
        0 & -4 & 3 & \BAR & 4 \\
        0 & 0  & 5 & \BAR & 10
        \end{gmatrix}
    \end{aligned}
\end{equation*}

\end{minipage}

\footnotetext{In der Matixform werden nur die Koeffizienten geschrieben, das spart Zeit und man verwechselt keine Variablen}

\pagebreak

\section{Lösungsmengen linearer Gleichungssysteme}

\begin{satz}
    Ein LGS hat entweder \textbf{genau eine} oder \textbf{keine} Lösung oder \textbf{unendlich viele} Lösungen.
\end{satz}
\noindent\rule{\textwidth}{1pt}

\textit{Beispiele:}

\begin{minipage}[t]{0.5\textwidth}
    (1) eine Lösung
    \begin{equation*}
        \begin{aligned}
            \begin{gmatrix}[p]
            1  & 2  & 1 & \BAR & 9 \\
            -2  & -1  & 5 & \BAR & 5 \\
            1  & -1  & 3 & \BAR & 4 
            \rowops
            \mult{0}{\cdot2}\add{0}{1}
            \mult{0}{(-1)}\add{0}{1}
            
            \end{gmatrix}
            
            \par\noindent\rule{0.5\textwidth}{0.4pt}
            
            \begin{gmatrix}[p]
            1  & 2  & 1 & \BAR & 9 \\
            0 & 3 & 7 & \BAR & 23 \\
            0 & -3 & 2 & \BAR & -5
            \rowops
            \add{1}{2}
            \end{gmatrix}
            
            \par\noindent\rule{0.5\textwidth}{0.4pt}
            
            \begin{gmatrix}[p]
            1  & 2  & 1 & \BAR & 9 \\
            0 & 3 & 7 & \BAR & 23 \\
            0 & 0  & 9 & \BAR & 18
            \end{gmatrix}
        \end{aligned}
    \end{equation*}
    
    Also: $x_3 = 2, \ x_2 = 3, \ x_1 = 1$ \\
    $\mathbb{L}$ = \{$(1; \ 3; \ 2)$\} \\

    (2) keine Lösung
    \begin{equation*}
        \begin{aligned}
            \begin{gmatrix}[p]
            1  & 2  & 3 & \BAR & 4 \\
            1  & 2  & -2 & \BAR & 0 \\
            -3  & -6  & 6 & \BAR & -1 
            \rowops
            \mult{0}{(-1)}\add{0}{1}
            \mult{0}{(3)}\add{0}{2}
            
            \end{gmatrix}
            
            \par\noindent\rule{0.5\textwidth}{0.4pt}
            
            \begin{gmatrix}[p]
            1  & 2  & 3 & \BAR & 4 \\
            0 & 0 & -5 & \BAR & -4 \\
            0 & 0 & 15 & \BAR & 11
            \rowops
            \end{gmatrix}
        \end{aligned}
    \end{equation*}
    
    $x_3$ hat zwei verschiedene Lösungen \\ $\rightarrow$ Widerspruch\\
    $\Rightarrow$ keine Lösung bzw. $\mathbb{L}$ = \{\} \\
    
\end{minipage}
\begin{minipage}[t]{0.5\textwidth}
    (3) unendlich viele Lösungen
    \begin{equation*}
    \begin{aligned}
        \begin{gmatrix}[p]
        2 \ & 6  & -3 & \BAR & -6 \\
        4 \ & 3  & 3 & \BAR & 6 \\
        4  & -3  & 9 & \BAR & 18 
        \rowops
        \mult{0}{(-2)}\add{0}{1}\add{0}{2}
            
        \end{gmatrix}
            
        \par\noindent\rule{0.5\textwidth}{0.4pt}
            
        \begin{gmatrix}[p]
        2  & 6  & -3 & \BAR & -6 \\
        0 & -9 & 9 & \BAR & 4 \\
        0 & -15 & 15 & \BAR & 30 
        \rowops
        \mult{1}{(-5)}\mult{2}{\cdot3}\add{1}{2}
        \end{gmatrix}
            
        \par\noindent\rule{0.5\textwidth}{0.4pt}
            
        \begin{gmatrix}[p]
        2 & 6 & -3 & \BAR & -6 \\
        0 & -9 & 9 & \BAR & 4 \\
        0 & 0  & 0 & \BAR & 0
        \end{gmatrix}
    \end{aligned}
    \end{equation*}
    
Also: $x_3 = t$ \quad , $t \in \mathbb{R}$ \\
Damit: $x_2 = -2-t$, $x_1 = 3 + \frac{9}{2}t$
    
$\mathbb{L}$ = $\left\{\left(3 + \frac{9}{2}t; \ -2-t; \ t\right) \middle| t \in \mathbb{R}\right\}$

\end{minipage}

\ \\
\ \\
\ \\


\section{LGS mit Parametern auf der rechten Seite}

\textit{Beispiele:}

\begin{minipage}[t]{0.5\textwidth}
    (1) Bestimmen sie den Wert von $r$ so, \\ dass das LGS die die angegebene Lösung \\ besitzt. \\
    \begin{equation*}
        \begin{aligned}
            \begin{gmatrix}[p]
            2 & -2 & 1 & \BAR & -3 \\
            3 & -1 & 1 & \BAR & r-2 \\
            4 & 1 & -1 & \BAR & 3,5r
    \end{gmatrix}
        \end{aligned}
    \end{equation*}
    
    Lösung: $(1; \ 2; \ -1)$ \\
    
    (1) $-3 = -3$\\
    (2) $0 = r-2 \Leftrightarrow r = 2$\\
    (3) $7 = 3,5r \ \Leftrightarrow r = 2$
\end{minipage}
\begin{minipage}[t]{0.5\textwidth}
    (2) Untersuchen Sie, ob es Werte von $r$ gibt, sodass das LGS keine oder unendlich viele Lösungen besitzt. Geben Sie im Fall unendlich vieler Lösungen die Lösungsmenge an. \\
    \begin{equation*}
    \begin{aligned}
        \begin{gmatrix}[p]
        2 & 1 & 1 & \BAR & 1 \\
        0 & -1 & 1 & \BAR & 2 \\
        0 & 3 & -3 & \BAR & r
        \rowops
        \mult{1}{\cdot 3}\add{1}{2}
        \end{gmatrix}
        
        \par\noindent\rule{0.5\textwidth}{0.4pt}
                
        \begin{gmatrix}[p]
            2 & 1 & 1 & \BAR & 1 \\
            0 & -1 & 1 & \BAR & 2 \\
            0 & 0 & 0 & \BAR & r + 6 
        \end{gmatrix}
    \end{aligned}
    \end{equation*}

    Aus $0 = r + 6$ folgt: \\
    für $r \neq 6$ besitzt das LGS keine Lösung, also $\mathbb{L} = \{\}$ \\
    für $r = -6$ besitzt das LGS unendlich viele Lösungen:\\
    
    Setze $x_3 = t \qquad ,t \in \mathbb{R}$ \\
    Damit= $\mathbb{L}_{-6} = \left\{\left(-t + \frac{3}{2}; \ t - 2 ; \ t\right) \middle| t \in \mathbb{R}\right\}$
\end{minipage}

\chapter{Geraden und Ebenen}
\section{Wiederholung: Vektoren im Raum}

\subsection{Vektoren im Raum $\mathbb{R}^3$}

\begin{definition}
    Vektoren im Raum sind Zahlentripel, z. B. $$\vec{v} = \left(\begin{array}{c} x_1 \\ x_2 \\ x_3 \end{array}\right)$$ Wir geben Vektoren in Spaltenform an. Ein Vektor beschreibt eine Verschiebung im Raum. Ein Vektor lässt sich als Pfeil im Raum darstellen. Alle Pfeile, die zu einem Vektor gehören, sind zueinander parallel, gleichlang und gleich gerichtet.
\end{definition}

\noindent\rule{\textwidth}{1pt}

\textit{Beispiel:} Der Vektor $\vec{v} = \left(\begin{array}{c} 2 \\ -2 \\ 5 \end{array}\right)$ verschiebt einen Ausgangspunkt $A$ um 2 Einheiten in $x_1$-Achsenrichtung, um -2 Einheiten in $x_2$-Achsenrichtung und um 5 Einheiten in $x_3$-Achsenrichtung.

\subsection{Allgemeine Darstellung}

Allgemein wählen wir \textbf{\textendash{}} bezogen auf ein kartesisches Koordinatensystem \textbf{\textendash{}} einen beliebigen Anfangspunkt. Dann bewegen wir uns je nach Vorzeichen der Koordinaten \\[5pt]
um $\left|a_1\right|$ in Richtung (Gegenrichtung) der $x_1$-Achse, um $\left|a_2\right|$ in Richtung (Gegenrichtung) der $x_2$-Achse und um $\left|a_3\right|$ in Richtung (Gegenrichtung) der $x_3$-Achse und kommen zum Endpunkt es Pfeils.

\subsection{Berechnung eines Vektors: Anfangspunkt $A(a_1|a_2|a_3)$ und Endpunkt $B(b_1|b_2|b_3)$}

Wir berechnen: $\vv{v} = \vv{AB} = \left(\begin{array}{c} b_1 \\ b_2 \\ b_3 \end{array}\right) - \left(\begin{array}{c} a_1 \\ a_2 \\ a_3 \end{array}\right) = \left(\begin{array}{c} b_1 - a_1 \\ b_2 - a_2 \\ b_3 - a_3 \end{array}\right) = \left(\begin{array}{c} v_1 \\ v_2 \\ v_3 \end{array}\right)$

\begin{definition}
    Der \textbf{Ortsvektor} des Punktes $A$ geht vom Ursprung zum Punkt $A$ und hat die gleichen Koordinaten wie der Punkt $A$, d. h. $$\vv{OA} = \left(\begin{array}{c} a_1 \\ a_2 \\ a_3 \end{array}\right)$$
\end{definition}

\subsection{Verschiebung eines Dreiecks}

Gegeben sind die Punkte $A(1 \ |-1| \ 2), B(-2| \ 2 \ |-4), C(7 \ | \ 2 \ | \ 0) \text{ und } D(2 \ | \ 2 \ | \ 6)$.

Der Vektor $\vv{v} = \vv{AD} = \left(\begin{array}{c} 2 - 1 \\ 2 + 1 \\ 6 - 2 \end{array}\right) = \left(\begin{array}{c} 1 \\ 3 \\ 4 \end{array}\right)$ verschiebt das Dreieck $ABC$ auf das Dreieck $DEF$. Berechne die Koordinaten der fehlenden Punkte mit Hilfe der Vektorrechnung.

$\vv{OE} = \vv{OB} + \vv{AD} = \left(\begin{array}{c} -2 \\ 2 \\ -4 \end{array}\right) + \left(\begin{array}{c} 1 \\ 3 \\ 4 \end{array}\right) = \left(\begin{array}{c} -1 \\ 5 \\ 0 \end{array}\right) \Rightarrow E(-1| \ 5\ | \ 0)$ \qquad ($F$ analog zu $E$)

\subsection{Gegenvektor}

\begin{minipage}{0.65\textwidth}
    \begin{definition}
    Der \textbf{Gegenvektor} von \textbf{$\vv{a} = \vv{AB}$} ist \textbf{$-\vv{a} = -\vv{AB} = \vv{BA}$}.
    \end{definition}

    \textit{Beispiel:} $A(1 \ | -1 | \ 2), B(-2| \ 2 \ | -4)$ \\[3pt]

    $\vv{a} = \vv{AB} = \left(\begin{array}{c} -2 - 1 \\ 2 + 1 \\ -4 - 2 \end{array}\right) = \left(\begin{array}{c} -3 \\ 3 \\ -6 \end{array}\right); -\vv{a} = \left(\begin{array}{c} 3 \\ -3 \\ 6 \end{array}\right)$
\end{minipage}
\begin{minipage}{0.1\textwidth}
  \  
\end{minipage}
\begin{minipage}{0.25\textwidth}
    \begin{tikzpicture}
      \def\ul{0.52}
      \def\R{4}
      \def\ang{45}
      \draw[vector,black]
        (0,0) (0,0) node[below] {$A$} -- (\ang:\R) node[above] {$B$};
      \draw[vector,black,shift={(\ang-90:1)}]
        (\ang:\R) node[above] {$B$} -- (0,0) node[below] {$A$};
      \draw (0.75,1.5) node[above] {$\vv{a} = \vv{AB}$};
      \draw (3.2,0.25) node[above] {$-\vv{a} = \vv{BA}$};
    \end{tikzpicture}
\end{minipage}

\begin{definition}[Nullvektor]
    Mit $\vv{0}$ bezeichnen wir den Nullvektor, d. h. den Vektor mit der Länge 0: $\vv{a} + (-\vv{a}) = \vv{0}$. 
\end{definition}

\subsection{Die Addition von Vektoren}

\begin{definition}
    Die Summe zweier Vektoren $\vv{a}$ und $\vv{b}$ ist wiederum ein Vektor und entspricht der Hintereinanderausführung (Verkettung) zweier Verschiebungen.
\end{definition}

\begin{minipage}{0.5\textwidth}
    Wir erhalten den \textbf{Summenvektor $\vv{a} + \vv{b}$}, indem wir den Anfang des Vektors $\vv{b}$ an die Spitze des Vektors $\vv{a}$ verschieben. Der Pfeil des Summenvektors $\vv{c} = \vv{a} + \vv{b}$ beginnt im Fußpunkt des Pfeils von $\vv{a}$ und endet an der Spitze des Pfeils von $\vv{b}$.
\end{minipage}
\begin{minipage}{0.05\textwidth}
    \
\end{minipage}
\begin{minipage}{0.4\textwidth}
    \begin{tikzpicture}[scale=0.85]
        \draw[vector, black] (0,0) -- (1,-1);
        \draw (0.5,-0.5) node[above] {$\vv{a}$};
    
        \draw[vector, black] (0,-3) -- (1.5,-1.5);
        \draw (0.65, -2.5) node[right] {$\vv{b}$};
    
        \draw[vector, black] (2,-2) -- (3,-3) node[text width=3.5cm, align=center, below] {Fuß von $\vv{b}$ an der Spitze von $\vv{a}$};
        \draw (2.5,-2.5) node[above] {$\vv{a}$};
        \draw[vector, black] (3,-3) -- (4.5,-1.5);
        \draw (3.75, -2.25) node[above] {$\vv{b}$};
    
        \draw[vector, black] (6,-2) -- (7,-3);
        \draw (7.2, -3.3) node[text width=3.5cm, align=center, below] {$\vv{a} + \vv{b} = \vv{c}$};
        \draw (6.5,-2.6) node[left] {$\vv{a}$};
        \draw[vector, black] (7,-3) -- (8.5,-1.5);
        \draw (7.6, -2.5) node[right] {$\vv{b}$};
        \draw[vector, black] (6,-2) -- (8.5,-1.5);
        \draw (7.1, -1.75) node[above] {$\vv{c}$};
        
    \end{tikzpicture}
\end{minipage}

Zeichnerisch gewinnen wir die Summe als Diagonalvektor des von $\vv{a}$ und $\vv{b}$ aufgespannten Parallelogramms.
    
Vektoren werden addiert, indem wir die einander entsprechenden Koordinaten addieren.

\subsection{Die Subtraktion von Vektoren}

\begin{minipage}{0.7\textwidth}
    Werden die Vektoren $\vv{a}$ und $\vv{b}$ durch zwei Pfeile mit \textbf{gleichem} Anfangspunkt dargestellt, dann entspricht $\vv{b} - \vv{a}$ dem Pfeil, der vom Endpunkt von $\vv{a}$ zum Endpunkt von $\vv{b}$ führt. Die Subtraktion eines Vektors entspricht der Addition des Gegenvektors.
\end{minipage}
\begin{minipage}{0.05\textwidth}
    \ 
\end{minipage}
\begin{minipage}{0.25\textwidth}
    \begin{tikzpicture}
        \draw[vector, black] (0,-2) -- (3,-1.75);
        \draw (1.5, -1.875) node[below] {$\vv{a}$};
        \draw[vector, black] (0,-2) -- (0.75,0);
        \draw (0.375, -1) node[left] {$\vv{b}$};
        \draw[vector, black] (3,-1.75) -- (0.75,0);
        \draw (1.6, -0.5) node[right] {$\vv{b} - \vv{a}$};
        
    \end{tikzpicture}
\end{minipage}

\subsection{Gesetze der Vektoraddition}

\begin{satz}[Kommutativgesetz]
    Bei der Vektoraddition dürfen wir die Summanden vertauschen, ohne dass sich das Ergebnis, die Vektorsumme ändert: $$\vv{a} + \vv{b} = \vv{b} + \vv{a}$$
\end{satz}
\begin{satz}[Assoziativgesetz]
    Bei Vektorsummen dürfen wir Klammern umsetzen oder weglassen, ohne dass sich am Ergebnis etwas ändert: $$(\vv{a} + \vv{b}) + \vv{c} = \vv{a} + (\vv{b} + \vv{c}) = \vv{a} + \vv{b} + \vv{c}$$
\end{satz}

\subsection{Geometrische Darstellung der Multiplikation eines Vektors mit einer reellen Zahl (Skalarmultiplikation)}

\begin{minipage}{0.7\textwidth}
    Sei $\vv{a} \in \mathbb{R}^3$ und $r \in \mathbb{R}$. Werden die Vektoren $\vv{a}$ und $r\cdot \vv{a}$  durch Pfeile im Raum dargestellt, dann gilt: Der zu $r\cdot \vv{a}$ gehörige Pfeil hat die $|r|$-fache Länge des zu $\vv{a}$ gehörigen Pfeils. $r$ heißt Skalar.

    Ist $\vv{a} \neq \vv{0}$, dann sind die beiden Pfeile gleichsinnig parallel, falls $r>0$, und ungleichsinnig parallel, falls $r<0$.
\end{minipage}
\begin{minipage}{0.05\textwidth}
    \ 
\end{minipage}
\begin{minipage}{0.25\textwidth}
    \begin{tikzpicture}
        \draw[vector, black] (0,-2) -- (1,-1);
        \draw (0.55,-1.3) node[left] {$\vv{a}$};
        \draw[vector, black] (1,-2) -- (3,0);
        \draw (2,-1) node[right] {$r \cdot \vv{a}$};
        
    \end{tikzpicture}
\end{minipage}

\begin{definition}
    Zwei Vektoren ($\neq \vv{0}$) heißen \textbf{kollinear bzw. parallel}, wenn sie Vielfache voneinander sind.
\end{definition}
Es gilt das \textbf{Distributivgesetz}: $r \cdot (\vv{a} + \vv{b}) = r \cdot \vv{a} + r \cdot \vv{b}$.

\section{Geraden im Raum}

\begin{minipage}{0.65\textwidth}
    \begin{definition}
        Jede Gleichung lässt sich durch eine Gleichung der Form $$g: \ \vv{x} = \vv{p} + t\vv{u} \qquad ,t \in \mathbb{R}$$ darstellen. $\vv{p}$ heißt \textbf{Stützvektor}, $\vv{u}$ heißt \textbf{Richtungsvektor} und $\vv{x}$ ist \textbf{Orstvektor} aller Punkte auf der Gerade.
    \end{definition}
\end{minipage}
\begin{minipage}{0.05\textwidth}
    \ 
\end{minipage}
\begin{minipage}{0.3\textwidth}
    \begin{tikzpicture}[scale=0.8]
    \begin{axis}[grid, clip = false,
    x={(-0.7071cm,-0.7071cm)},    
    y={(1cm,0.0cm)}, 
    z={(0cm,1cm)},
    axis lines=center,
    font=\footnotesize,
    xmin=0, xmax=2.5, ymin=0,ymax=3,zmin=0,zmax=3,
    xtick={0,...,0},ytick={0,...,0},ztick={0,...,0},
    xlabel=\normalsize$x_1$,ylabel=\normalsize$x_2$,zlabel=\normalsize$x_3$]
    
    \addplot3+[thick, mark=,magenta,->-=1] coordinates {(0,0,0) (3,0,3)};
    \draw (1.6,0,1.7) node[right,magenta]{$\vv{OP} = \vv{p}$};
    \addplot3+[thick, mark=,red] coordinates {(3,0,3) (0,2,3)} node[below, pos=0.95] {$g$};
    \addplot3+[thick, mark=,red] coordinates {(3,0,3) (4,-0.6,3)};
    \addplot3+[thick, mark=,blue,->-=1] coordinates {(3,0,3) (0,0,1.96)} node[above, pos=.5] {$\vv{u}$};
    
    \draw (0,0,0) node[below, magenta] {$O$} {};
    \draw (3,0,3) node[above, magenta] {$P$} {};
    
    \end{axis}
    \end{tikzpicture}
\end{minipage}

\noindent\rule{\textwidth}{1pt}

\textit{Beispiel:}

(1) Liegen die Punkte $A(2 \ | \ 3 \ | \ 2), B(1 \ | \ 1 \ | \ 1)$ und $C(3 \ | -1 | \ 1)$ auf einer Geraden?

\underline{Aufstellen einer Geradengleichung durch zwei Punkte}

$g_{AB}: \vv{x} = \underbrace{\left(\begin{array}{c}  2 \\ 3 \\ 2 \end{array}\right)}_{\vv{OA} = \vv{a}} + t \underbrace{\left(\begin{array}{c} -1 \\ -2 \\ -1 \end{array}\right)}_{\vv{AB}} \qquad, t \in \mathbb{R}$ 


\pagebreak

\underline{Punktprobe: $C \in g_{AB}$}

\begin{minipage}{0.5\textwidth}
    $\left(\begin{array}{c}  3 \\ -1 \\ 1 \end{array}\right) = \left(\begin{array}{c}  2 \\ 3 \\ 2 \end{array}\right) + t \left(\begin{array}{c} -1 \\ -2 \\ -1 \end{array}\right)$
\end{minipage}
\begin{minipage}{0.5\textwidth}
    \begin{align*}
        3 & = 2 - t &\Rightarrow & t = -1 \\
        -1 & = 3 - 2t &\Rightarrow & t = 2 \\
        1 & = 2 - t &\Rightarrow & t = 1
    \end{align*}

    $\Rightarrow$ keine eindeutige Lösung $\Rightarrow C \not\in g_{AB}$
\end{minipage}\\

(2) Gegeben sind die Punkte $P_r (1 \ | \ 2 - r \ | \ 3r)$ und $Q_s(s+2 \ | \ 1-s \ | \ s+1)$. 

a) Begründen Sie, dass alle Punkte $P_r$ bzw. $Q_s$  auf einer Geraden $p$ bzw. $q$ lieben.

b) Untersuchen Sie die gegenseitige Lage von $p$ und $q$. \\

a) Die Ortsvektoren der Punkte $P_r$ und $Q_s$ erfüllen genau die Gleichung\footnote{hier wird der Punkt (mit Parameter) quasi \glqq auseinandergezogen\grqq{}}:

$p: \vv{x} = \left(\begin{array}{c}  1 \\ 2 \\ 0 \end{array}\right) + r \left(\begin{array}{c}  0 \\ -1 \\ 3 \end{array}\right) \qquad ,r \in \mathbb{R}$

$q: \vv{x} = \left(\begin{array}{c}  2 \\ 1 \\ 1 \end{array}\right) + s \left(\begin{array}{c}  1 \\ -1 \\ 1 \end{array}\right) \qquad ,s \in \mathbb{R}$ \\

b) Es gilt: $\vv{u_p} \neq k \cdot \vv{u_q} \Rightarrow p \nparallel q$

Weiter: $P_0 (1 \ | \ 2 \ | \ 0) \in p \quad \land \quad Q_1 (1 \ | \ 2 \ | \ 0) \in q$

Also: Die beiden Geraden schneiden sich in $S (1 \ | \ 2 \ | \ 0)$


\pagebreak

\subsection{Wiederholung: Gegenseitige Lage von Gerade}

Gegeben sind zwei Geraden $g$ und $h$ mit $g: \ \vv{x} = \vv{p} + r \vv{u}$ und $h: \ \vv{x} = \vv{q} + r \vv{v}$, wobei $r,s \in \mathbb{R}$

\begin{tikzpicture}[node distance=2cm]
        \node (pro1) [process, text width=4cm] {Sind die Richtungsvektoren $\vv{u}$ und $\vv{v}$ Vielfache voneinander?};
        \node (pro2a) [process, text width=2cm, below of=pro1, xshift = -5cm] {$g \parallel h$};
        \node (pro2b) [process, text width=2cm, below of=pro1, xshift = 5cm] {$g \nparallel h$};
        \node (pro3a) [process, below of=pro2a, yshift= -1cm, text width=5.5cm] {Liegt der Punkt $P$ mit dem Ortsvektor $\vv{p}$ auf der Geraden $h$? ($\rightarrow$ Punktprobe)};
        \node (pro3b) [process, below of=pro2b, yshift= -1cm, text width=5.5cm] {Hat die Gleichung \\ $\vv{p} + r \vv{u} = \vv{q} + r \vv{v}$ eine einzige Lösung? ($\rightarrow$ LGS)};
        \node (pro4a) [process, text width=3cm, below of=pro3a, xshift= -2.25cm, yshift= -1cm] {g und h sind \textbf{identisch}.};
        \node (pro4b) [process, text width=3cm, below of=pro3a, xshift= 2.25cm, yshift= -1.35cm] {g und h sind zueinander \textbf{parallel}.};
        \node (pro4c) [process, text width=3cm, below of=pro3b, xshift= -2.25cm, yshift= -1.67cm] {g und h \textbf{schneiden} sich in einem Punkt $S$.};
        \node (pro4d) [process, text width=3cm, below of=pro3b, xshift= 2.25cm, yshift= -1.35cm] {g und h sind zueinander \textbf{windschief}.};
        \node (pro5) [process, text width=3cm, below of=pro4c,yshift=-1cm] {Bestimmung des Schnittpunktes $S$.};
    
    
        \draw [arrow] (pro1) -- node[anchor=south] {ja} (pro2a);
        \draw [arrow] (pro1) -- node[anchor=south west] {nein} (pro2b);
        \draw [arrow] (pro2a) -- (pro3a);
        \draw [arrow] (pro2b) -- (pro3b);
        \draw [arrow] (pro3a) -- node[left] {ja} (pro4a);
        \draw [arrow] (pro3a) -- node[right] {nein} (pro4b);
        \draw [arrow] (pro3b) -- node[left] {ja} (pro4c);
        \draw [arrow] (pro3b) -- node[right] {nein} (pro4d);
        \draw [arrow] (pro4c) -- (pro5);
    \end{tikzpicture}

%hässliche Lösung
\ \\
\ \\
\ \\
\ \\
\ \\
\ \\
\ \\
\ \\
\ \\
\ \\
\ \\


\section{Ebenen im Raum}

\begin{satz}
    $P$ sei ein Punkt mit \textbf{Ortsvektor} $\vv{p}$ (\textbf{Stützvektor}); $\vv{u}$ und $\vv{v}$ seinen zwei linear unabhängige (d. h. nicht parallele) Vektoren (\textbf{Spannvektoren}). Alle Punkte $X$ mit dem Ortsvektor $\vv{x}$ mit $\vv{x} = \vv{p} + r \vv{u} + s \vv{v}$ mit $r, s \in \mathbb{R}$ bilden eine Ebene durch den Punkt. 
    $$E: \ \vv{x} = \vv{p} + r \vv{u} + s \vv{v} \qquad (r, s \in \mathbb{R})$$
    heißt Parameterform der Ebene $E$.
    
\end{satz}
% All numbers must be in [0,1] to fall in the plane drawn
\def\t{0.9}
\def\s{0.04}
\def\tt{0.7}
\def\kk{0.7}
\def\ttt{0.324}
\def\sss{0.6666}
\def\tttt{0.4}
\def\ssss{0.4}
\def\ttttt{0.85}
\def\sssss{0.7}
\def\tttttt{0.7}
\def\ssssss{0.2}

\begin{tikzpicture}[x={(-0.7071cm,-0.7071cm)}, y={(1cm,0.0cm)}, z={(0cm,1cm)}, line cap=round, line join=round,scale = 2.35]
	%Coordinates
	%Plane Vertex Points
	\coordinate (x1) at (-1,2,1);
	\coordinate (x2) at (2,2,5);
	\coordinate (x3) at (2,-2,5);
	\coordinate (x4) at (-1,-2,1);
	%Vectors Parallel to Plane
	\coordinate (n1) at ($(x2) - (x1)$);
	\coordinate (n2) at ($(x2) - (x3)$); 
	%Points on Plane
	\coordinate (x5) at ($(x1) + \s*(n1) - \t*(n2)$);
	\node[outer sep = 1pt, inner sep = 1pt] (x6) at ($(x1) + \kk*(n1) - \tt*(n2)$) {};
	\coordinate (x7) at ($(x1) + \sss*(n1) - \ttt*(n2)$);
	%Beginning of Axis
	\coordinate (O) at (0,0,0);
	
	%Axis 	
	\draw[-latex] (-2,0,0) -- (2,0,0) node[pos = 1.05] {$x_1$};
	\draw[-latex] (0,-3,0) -- (0,3,0) node[pos = 1.05] {$x_2$};
	\draw[-latex] (0,0,0) -- (0,0,5) node[pos = 1.05] {$x_3$};
	\draw[draw=black, fill=black] (O) circle (0.5pt);
	
	%Point on Plane
	\draw[-latex, thick] (O) -- (x6) node[pos=0.45, left] {$\vb{\vv{p}}$};

    %Dashed Lines
    \node[outer sep = 1pt, inner sep = 1pt] (x11) at ($(x1) + \ssssss*(n1) - \tttttt*(n2)$) {};
    \draw[dashed, thick] (O) -- (x11);
    
	%Plane
	\path[draw=black, fill=black!20, thick, opacity = 0.8] (x1) -- (x2) -- (x3) -- (x4) -- (x1);
	\node[shift={(-0.3,0)}] at (x3) {$E$};
	%Point on Plane	
	\draw[draw=black, fill=black] (x6) node[thick, cross] {} node[above right] {${P}$};
    \draw[draw=black, fill=black] (x11) node[thick, cross] {} node[right] {${X}$};

    %Vectors on Plane
    \node[outer sep = 1pt, inner sep = 1pt] (x8) at ($(x1) + \ssss*(n1) - \tttt*(n2)$) {};
    \draw[-latex, thick,draw=black] (x6) -- (x8) node[pos=0.6, above] {$\vb{\vv{u}}$};
    
    \node[outer sep = 1pt, inner sep = 1pt] (x9) at ($(x1) + \sssss*(n1) - \ttttt*(n2)$) {};
    \draw[-latex, thick,draw=black] (x6) -- (x9) node[pos=0.6, above] {$\vb{\vv{v}}$};

    \node[outer sep = 1pt, inner sep = 1pt] (x10) at ($(x1) + \ssssss*(n1) - \tttttt*(n2)$) {};
    \draw[-latex, thick,draw=black] (x6) -- (x10) node[pos=0.6, left, shift={(0.1,-0.1)}] {$\vb{r\cdot\vv{u} + s\cdot\vv{v}}$};
	
	%Z-axis Section
	\draw[draw=black, fill=black] (x7) circle (0.5pt);
	\draw (x7) -- (0,0,4.5);
	
%	%Random Point
%	\draw[-latex, thick] (O) -- (P) node[pos=0.45, shift={(0.1,0.3)}] {$\vb{r}$};
%	\draw[draw=black, fill=black] (P) circle (1pt) node[above right] {$\mathrm{P}$};
	
\end{tikzpicture}


\pagebreak

Eine Ebene ist eindeutig definiert durch:
\begin{itemize}
    \item Durch einen Punkt und zwei linear unabhängige Vektoren, den Spannvektoren.
    \item Durch zwei sich schneidende Geraden.
    \item Durch drei Punkte, die nicht alle drei auf einer Geraden liegen.
    \item Durch zwei parallele, nicht identische Geraden.
    \item Durch eine Gerade $g$ und einen Punkt, der nicht auf $g$ liegt.
\end{itemize}

\textit{Vorgehensweise: Bestimmen einer Ebene mit drei gegebenen Punkten}

Gegeben sind die Punkte $P(1 \ | \ 3 \ | 2)$, $Q(2 \ | -3 \ | \ 5)$ und $R(1 \ | -3 \ | -4)$ einer Ebene $E$. Stellen sie eine Parameterdarstellung von $E$ auf.
\begin{itemize}
    \item Wähle einen Punkt der drei Punkte als Aufpunkt aus, z. B. $P(1 \ | \ 3 \ | 2)$. Sein Ortsvektor wird der Stützvektor der Ebene $E$.
    \item Berechne aus den drei gegebenen Punkten zwei linear unabhängige Vektoren, z. B. $\vv{PQ}$ und $\vv{PR}$. Diese zwei Vektoren sind nun Spannvektoren der Ebene $E$.
    \item Stelle nun eine Parametergleichung der Ebene E auf.
\end{itemize}

\textit{Besondere Ebenen: $x_1x_2$-Ebene, $x_2x_3$-Ebene und $x_1x_3$-Ebene}

\begin{equation*}
    \begin{aligned}
        x_1x_2\text{-Ebene: } & E_{12}: & \vv{x} = \left(\begin{array}{c}  0 \\ 0\\ 0 \end{array}\right) + r \left(\begin{array}{c}  1 \\ 0 \\ 0 \end{array}\right) + s \left(\begin{array}{c}  0 \\ 1 \\ 0 \end{array}\right) \qquad ; r,s \in \mathbb{R} \\
        x_2x_3\text{-Ebene: } & E_{23}: & \vv{x} = \left(\begin{array}{c}  0 \\ 0\\ 0 \end{array}\right) + r \left(\begin{array}{c}  0 \\ 1 \\ 0 \end{array}\right) + s \left(\begin{array}{c}  0 \\ 0 \\ 1 \end{array}\right) \qquad ; r,s \in \mathbb{R} \\
        x_1x_3\text{-Ebene: } & E_{13}: & \vv{x} = \left(\begin{array}{c}  0 \\ 0\\ 0 \end{array}\right) + r \left(\begin{array}{c}  1 \\ 0 \\ 0 \end{array}\right) + s \left(\begin{array}{c}  0 \\ 0 \\ 1 \end{array}\right) \qquad ; r,s \in \mathbb{R} 
    \end{aligned}
\end{equation*}


\pagebreak

\textit{Beispiel: Punktprobe}

Liegt der Punkt $P$ mit $P(8 \ | \ 0 \ | \ 6)$ in der Ebene $E$ mit

$E: \ \vv{x} = \left(\begin{array}{c}  1 \\ 0\\ -1 \end{array}\right) + r \left(\begin{array}{c}  2 \\ 1 \\ 3 \end{array}\right) + s \left(\begin{array}{c}  3 \\ -2 \\ 1 \end{array}\right)$ ?

$P$ in $E$:
\begin{gather*}
    \left(\begin{array}{c}  8 \\ 0\\ 6 \end{array}\right) = \left(\begin{array}{c}  1 \\ 0\\ -1 \end{array}\right) + r \left(\begin{array}{c}  2 \\ 1 \\ 3 \end{array}\right) + s \left(\begin{array}{c}  3 \\ -2 \\ 1 \end{array}\right)
\end{gather*}

LGS: 

$\begin{gmatrix}
        2r & + & 3s & = & 7 \\
        r & - & 2s & = & 0 \\
        3r & + & s & = & 7
        \rowops
        \mult{1}{(-2)}\add{1}{0}
        \end{gmatrix}$
        
        \par\noindent\rule{0.65\textwidth}{0.4pt}
        
        $\begin{gmatrix}
        \text{Also:} & \quad & \ & \ & \ & 7s & = & 7 \\
        & \quad & \ & \Longleftrightarrow & \ & s & = & 1 \\
        & \quad & \ & \ & \ & r & = & 2 
    \end{gmatrix}$

Probe in letzter Zeile:
\begin{gather*}
    3 \cdot 2 + 1 = 7
\end{gather*}
Also: $P \in E$

\section{Zueinander orthogonale Vektoren: Skalarprodukt}

\begin{definition}
    Zwei Vektoren $\vv{a}$, $\vv{b} (\neq \vv{0})$  heißen zueinander \textbf{orthogonal}, wenn ihre zugehörigen Pfeile mit gleichem Anfangspunkt ebenfalls zueinadner orthogonal (d. h. senkrecht) sind. In Zeichen: 
    $$\vv{a} \perp \vv{b}$$
\end{definition}

\begin{definition}
    Unter dem \textbf{Skalarprodukt} zweier Vektoren $\vv{a} = \left(\begin{array}{c}  a_1 \\ a_2 \\ a_3 \end{array}\right)$ und \\ $\vv{b} = \left(\begin{array}{c}  b_1 \\ b_2 \\ b_3 \end{array}\right)$ versteht man die reelle Zahl 
    $$a_1 \cdot b_1 + a_2 \cdot b_2 + a_3 \cdot b_3$$
    Das Skalarprodukt der Vektoren $\vv{a}$ und $\vv{b}$ wird mit $\vv{a} \circ \vv{b}$ bezeichnet:
    $$\vv{a} \circ \vv{b} = \left(\begin{array}{c}  a_1 \\ a_2 \\ a_3 \end{array}\right) \circ \left(\begin{array}{c}  b_1 \\ b_2 \\ b_3 \end{array}\right) = a_1 \cdot b_1 + a_2 \cdot b_2 + a_3 \cdot b_3$$
\end{definition}

\begin{satz}[Orthogonalitätskriterium]
    Zwei Vektoren $\vv{a}$, $\vv{b} (\neq \vv{0})$ sind genau dann zueinander orthogonal, wenn für ihre Koordinaten gilt:
    $$a_1 \cdot b_1 + a_2 \cdot b_2 + a_3 \cdot b_3 = 0$$
    das heißt:
    $$\vv{a} \perp \vv{b} \text{,wenn: } \vv{a} \circ \vv{b} = 0$$
\end{satz}

Gesetze für das Skalarprodukt: \\
Für alle Vektoren $\vv{u}, \vv{v}, \vv{w}$ und alle reellen Zahlen $s$ gilt:
\begin{itemize}
    \item Kommutativgesetz: $\vv{u} \circ \vv{v} = \vv{v} \circ \vv{u}$ 
    \item Gemisches Assoziativgesetz: $s(\vv{u} \circ \vv{v}) = (s \vv{u}) \circ \vv{v} = \vv{u} \circ (s \vv{v})$
    \item Distributivgesetz: $\vv{u} \circ (\vv{v} + \vv{w}) = \vv{u} \circ \vv{v} + \vv{u} \circ \vv{w}$
\end{itemize}

\noindent\rule{\textwidth}{1pt}

\textit{Beispiel:}

Die dreieckige Teilfläche eines Daches kann in einem Koordinatensystem (Einheit 1 m ) durch die Eckpunkte $A(7 \ | -2 \ | \ 6), B(2 \ | \ 6 \ | \ 5)$ und $C(1 \ | \ 4\ | \ 3)$ beschrieben werden. Prüfe nach, ob das Dreieck $ABC$ rechtwinklig ist.

$\vv{AB} = \left(\begin{array}{c}  -5 \\ 8 \\ -1 \end{array}\right)$, $\vv{AC} = \left(\begin{array}{c}  -6 \\ 6 \\ -3 \end{array}\right)$, $\vv{BC} = \left(\begin{array}{c}  -1 \\ -2\\ -2 \end{array}\right)$


$\vv{AB} \circ \vv{AC} = \left(\begin{array}{c}  -5 \\ 8 \\ -1 \end{array}\right) \circ \left(\begin{array}{c}  -6 \\ 6 \\ -3 \end{array}\right) = -5 \cdot (-6) + 8 \cdot 6 + (-1) \cdot (-3) = 81 \neq 0 \Rightarrow \vv{AB} \not\perp \vv{AC} $
$\vv{AC} \circ \vv{BC} = \left(\begin{array}{c}  -6 \\ 6 \\ -3 \end{array}\right) \circ \left(\begin{array}{c}  -1 \\ -2\\ -2 \end{array}\right) = -6 \cdot (-1) + 6 \cdot (-2) + (-3) \cdot (-2) = 0 \Rightarrow \vv{AC} \perp \vv{BC}$

Das Dreieck $ABC$ ist folglich rechtwinklig.


\pagebreak

\section{Koordinatenform (=Koordinatengleichung) und Normalenform (=Normalengleichung) einer Ebene}

\begin{definition}
    Ein Normalenvektor $\vv{n}$ ist der Vektor, welcher orthogonal (rechtwinklig) zu einer Ebene $E$ (oder einer Geraden $g$) steht.
\end{definition}
Ein Normalenvektor, mit einem Stützvektor, ist ein weiterer Weg, um eine Ebene eindeutig zu definieren. \\

\begin{minipage}[t]{0.49\textwidth}
    \textbf{Normalenform}

    $$E: \ (\vv{x} - \vv{p}) \circ \vv{n} = 0$$

    $\vv{n}$ : (Ein) Normalenvektor \\
    $\vv{p}$ : Stützvektor \\
    \\[3.75pt]
    \noindent\rule{\textwidth}{1pt} \\
    \textit{NF in KF:} \\

    $E: \ \left[\vv{x} - \left(\begin{array}{c}  2 \\ 3 \\ 9 \end{array}\right)\right] \circ \left(\begin{array}{c}  4 \\ 3 \\ 0 \end{array}\right) = 0$ \\

    1. Möglichkeit: \textbf{Ausmultiplizieren} \\
    $\left(\begin{array}{c}  x_1 \\ x_2 \\ x_3 \end{array}\right) \circ \left(\begin{array}{c}  4 \\ 3 \\ 0 \end{array}\right) = \left(\begin{array}{c}  2 \\ 3 \\ 9 \end{array}\right) \circ \left(\begin{array}{c}  4 \\ 3 \\ 0 \end{array}\right)$ \\
    
    $E: \ 4x_1 + 3x_2 = 17$ \\

    2. Möglichkeit: \\
    $E: \ 4x_1 + 3x_2 = d$ \\

    Punktprobe mit $P(2 \ | \ 3 \ | \ 9)$: \\
    $4\cdot2 + 3\cdot3 = d \Longleftrightarrow d = 17$\\
    $E: \ 4x_1 + 3x_2 = 17$ \\
\end{minipage}
\begin{minipage}[t]{0.49\textwidth}
    \textbf{Koordinatenform}

    $$E: \ ax_1 + bx_2 + cx_3 = d$$
    
    $\left(\begin{array}{c}  a \\ b \\ c \end{array}\right)$ : (Ein) Normalenvektor \\
    \\
    \noindent\rule{\textwidth}{1pt} \\
    \textit{KF in NF:}\\
    
    $P(1 \ | \ 2 \ | -1)$, $E: \ x_1 - x_2 + 2x_3 = 3$ \\
    $P \not\in E$ ,da: $1 - 2 + 2 \cdot (-1) = -3  \neq 3$ \\

    $E: \ \left[\vv{x} - \vv{p}\right] \circ \left(\begin{array}{c}  1 \\ -1 \\ 2 \end{array}\right) = 0$ \footnotemark\\

    Wähle $Q(3 \ | \ 0 \ | \ 0)$, damit: \\ 

    $E: \ \left[\vv{x} - \left(\begin{array}{c}  3 \\ 0 \\ 0 \end{array}\right)\right] \circ \left(\begin{array}{c}  1 \\ -1 \\ 2 \end{array}\right) = 0$ \\
\end{minipage}

\footnotetext{Der Vektor ist hier ein Normalenvektor, gebildet aus den Koeffizienten der geg. Koordinatenform}

\begin{definition}
    Für die Vektoren $\vv{a} = \left(\begin{array}{c}  a_1 \\ a_2 \\ a_3 \end{array}\right)$ und $\vv{b} = \left(\begin{array}{c}  b_1 \\ b_2 \\ b_3 \end{array}\right)$ heißt $$\vv{a} \times \vv{b} = \left(\begin{array}{c}  a_2b_3 - a_3b_2 \\ a_3b_1 - a_1b_3 \\ a_1b_2 - a_2b_1 \end{array}\right)$$ (lies: \glqq  a Kreuz b\grqq{}) das \textbf{Vektorprodukt} von $\vv{a}$ und $\vv{b}$.
\end{definition}
\begin{satz}
    $\vv{a} \times \vv{b}$ ist orthogonal zu $\vv{a}$ und zu $\vv{b}$.
\end{satz}

Das Vektorprodukt eignet sich also, um einen Normalenvektor bei gegebener Parameterform auszurechnen: Der Normalenvektor ist das Kreuzprodukt der beiden Spannvektoren.

\subsection{Parameterform, Koordinatenform und Normalenform: Umformen}

\begin{tikzpicture}[node distance=2cm]
    \node (pro1) [process, text width=5.5cm] {\textbf{Parameterform} \\ $E: \ \vv{x} = \vv{OP} + r\cdot \vv{PQ} + s \cdot \vv{PR}$};
    \node (pro2a) [process, text width=5cm, left of=pro1, xshift = -4cm, yshift= -2cm] {Drei Punkte $P, Q$ und $R$ in $E$, die nicht auf einer Geraden liegen};
    \node (pro2b) [process, text width=5cm, right of=pro1, xshift = 4cm, yshift= -2cm] {$\vv{n} = \vv{PR} \times \vv{PR}$ mit \\ $\vv{n} = \left(\begin{array}{c}  a \\ b \\ c \end{array}\right), \vv{p} = \left(\begin{array}{c}  p_1 \\ p_2 \\ p_3 \end{array}\right)$};
    \node (pro3) [process, below of=pro1, text width=5.5cm, yshift= -2cm] {\textbf{Koordinatenform} \\ $E: \ ax_1 + bx_2 + cx_3 = d$ \\ mit $d = a\cdot p_1 + b \cdot p_2 + c \cdot p_3$};
    \node (pro4) [process, below of=pro1, text width=5.5cm, yshift= -5cm] {\textbf{Normalenform} \\ $E: \ (\vv{x} - \vv{p}) \circ \vv{n} = 0$};
    
    \draw [arrow] (pro2a) |- (pro1);
    \draw [arrow] (pro1) -| (pro2b);
    \draw [arrow] (pro2b) |- (pro3);
    \begin{scope}[transform canvas={xshift=1em}]
        \draw [arrow] (pro2b) |- (2.5,-7);
    \end{scope}
    \draw [arrow] (pro3) -| (pro2a);
    \begin{scope}[transform canvas={xshift=-1em}]
        \draw [arrow] (-2.5,-7) -| (pro2a);
    \end{scope}
    \begin{scope}[transform canvas={xshift=1em}]
        \draw [arrow] (pro3) -- (pro4);
    \end{scope}
    \begin{scope}[transform canvas={xshift=-1em}]
        \draw [arrow] (pro4) -- (pro3);
    \end{scope}
\end{tikzpicture}

\pagebreak

\subsection{Vergleich: Skalarprodukt und Vektorprodukt}

\begin{center}
    \begin{tabular} { | p{0.2\textwidth} | p{0.2\textwidth} | p{0.2\textwidth} | p{0.29\textwidth} | }
     \hline
      & Schreibweise & Das Ergebnis ist ein ... & Anwendung \\
     \hline
     Skalarprodukt  &  $\vv{a} \circ \vv{b}$\footnotemark &  ...Skalar (reelle Zahl) & Überprüfung der Orthogonalität von $\vv{a}$ und $\vv{b}$: $\vv{a} \circ \vv{b} = 0$  \\
     \hline
     Vektorprodukt (Kreuzprodukt) & $\vv{a} \times \vv{b}$ & ...Vektor & Suche einen Vektor $\vv{c}$ mit $\vv{c} \perp \vv{a}$ und $\vv{c} \perp \vv{b}$ \newline (Berechnung eines Normalenvektors einer Ebene bei gegebenen Spannvektoren)  \\
    \hline
    \end{tabular}
\end{center}

\footnotetext{früher: $\vv{a} \cdot \vv{b}$}

\section{Ebenen veranschaulichen}

\begin{definition}
    Die Schnittpunkte einer Ebene mit den Koordinatenachsen heißen \textbf{Spurpunkte}. 

    Sie werden mit $S_1(x_1 \ | \ 0 \ | \ 0)$, $S_2(0 \ | \ x_2 \ | \ 0)$ und $S_3(0 \ | \ 0 \ | \ x_3)$ bezeichnet. Die Schnittgeraden einer Ebene mit den Koordinatenebenen heißen \textbf{Spurgeraden}. Sie werden mit $s_{12}$, $s_{23}$ und $s_{13}$ bezeichnet.
\end{definition}


\begin{minipage}[t]{0.5\textwidth}
    Skizze:
    
    \begin{tikzpicture}[x={(-0.7071cm,-0.7071cm)}, y={(1cm,0.0cm)}, z={(0cm,1cm)}, line cap=round, line join=round,scale = 1]
	%Coordinates
	%Plane Vertex Points
	\coordinate (x1) at (2,0,0);
	\coordinate (x2) at (0,4,0);
	\coordinate (x3) at (0,0,3);
	
	\coordinate (O) at (0,0,0);
	
	%Axis 	
	\draw[-latex] (-1,0,0) -- (3,0,0) node[pos = 1.05] {$x_1$};
	\draw[-latex] (0,-1.5,0) -- (0,5,0) node[pos = 1.05] {$x_2$};
	\draw[-latex] (0,0,-0.5) -- (0,0,4) node[pos = 1.05] {$x_3$};
	\draw[draw=black, fill=black] (O) circle (0.5pt);
	
	
	%Plane
	\path[draw=black, fill=black!20, thick, opacity = 0.8] (x1) -- (x2) -- (x3) -- (x1);
	\node[shift={(0.7,-0.5)}] at (x2) {$E$};

        \draw (x1) node[thick, cross] {} node[below right] {$S_1$};
        \draw (x2) node[thick, cross] {} node[below right] {$S_2$};
        \draw (x3) node[thick, cross] {} node[above right] {$S_3$};

        \draw (x1) -- (x2) node[pos=0.5, below] {$s_{12}$};
        \draw (x2) -- (x3) node[pos=0.5, above] {$s_{23}$};
        \draw (x1) -- (x3) node[pos=0.5, left] {$s_{13}$};
	
	
    \end{tikzpicture}
\end{minipage}
\begin{minipage}{0.05\textwidth}
    \
\end{minipage}
\begin{minipage}[t]{0.45\textwidth}
    \textit{Beispiel:}

    $S_1(2 \ |  \ 0 \ | \ 0)$, $S_2(0 \ |  \ 4 \ | \ 0)$ und $S_3(0 \ |  \ 0 \ | \ 3)$

    $S_1$, $S_2$ und $S_3$ liegen in einer Ebene und $O \not\in E$; wähle d=1: \\

    $S_1: 2a = d$

    $S_2: 4b = d$

    $S_3: 3c = d$ \\
    Damit: $a= \frac{1}{2}; b = \frac{1}{4}; c = \frac{1}{3}$ \\
    Also: \\
    $E: \ \frac{1}{2}x_1 + \frac{1}{4}x_2 + \frac{1}{3}x_3 = 1$ \ \footnotemark

    $E: \ 6x_1 + 3x_2 + 4x_3 = 12$
\end{minipage}

\footnotetext{In Zukunft kann man auch direkt zu diesem Punkt springen, das Verfahren mit den Spurpunkten zur Ebene ist immer das Gleiche.}

\textit{Beispiel: Spurpunkte bestimmen bei gegebener Ebenengleichung}

$E: \ 2x_1 + 6x_2 + 4x_3 = 12$
\begin{align*}
    x_1\text{-Achse: } & x_2 + x_3 = 0 \\
    & 2x_1 = 12 \\
    & x_1 = 6 \Rightarrow S_1(6 \ | \ 0 \ | \ 0)
\end{align*}

$S_2$ und $S_3$ analog, aber allgemein keine Rechnung nötig! 

$S_2(0\ | \ 2 \ | \ 0)$; $S_3(0\ | \ 0 \ | \ 4)$ \\
    
\textbf{Ebenen veranschaulichen mit einer und zwei Varible(n)} 

\begin{minipage}[t]{0.475\textwidth}
\vspace{0pt}
    $E: \ 2x_1 + 6x_2 = 12$

    $S_1(6\ | \ 0 \ | \ 0)$; $S_2(0\ | \ 2 \ | \ 0)$

    $\rightarrow$ parallel zur $x_3$-Achse \\
    
    \begin{tikzpicture}[x={(-0.7071cm,-0.7071cm)}, y={(1cm,0.0cm)}, z={(0cm,1cm)}, line cap=round, line join=round,scale = 0.9]
	%Coordinates
	%Plane Vertex Points
	\coordinate (x1) at (6,0,0);
	\coordinate (x2) at (0,2,0);
	\coordinate (x3) at (6,0,4);
        \coordinate (x4) at (0,2,4);
 
	
	\coordinate (O) at (0,0,0);
	
	%Axis 	
	\draw[-latex] (-1,0,0) -- (7,0,0) node[pos = 1.05] {$x_1$};
	\draw[-latex] (0,-1.5,0) -- (0,3,0) node[pos = 1.05] {$x_2$};
	\draw[-latex] (0,0,-0.5) -- (0,0,4) node[pos = 1.05] {$x_3$};
	\draw[draw=black, fill=black] (O) circle (0.5pt);
	
	
	%Plane
	\path[draw=black, fill=black!20, thick, opacity = 0.8] (x3) -- (x1) -- (x2) -- (x4);
	\node[shift={(0.5,-0.7)}] at (x2) {$E$};

        \draw (x1) node[thick, cross] {} node[above left] {$S_1$};
        \draw (x2) node[thick, cross] {} node[above right] {$S_2$};

        \draw (x1) -- (x2) node[pos=0.5, below] {$s_{12}$};
	
    \end{tikzpicture}
\end{minipage}
\begin{minipage}{0.05\textwidth}
\end{minipage}
\begin{minipage}[t]{0.475\textwidth}
\vspace{0pt}
    $E: \ 4x_1  = 12$

    $S_1(3\ | \ 0 \ | \ 0)$

    $\rightarrow$ parallel zur $x_2x_3$-Ebene \\
    
    \begin{tikzpicture}[x={(-0.7071cm,-0.7071cm)}, y={(1cm,0.0cm)}, z={(0cm,1cm)}, line cap=round, line join=round,scale = 0.9]
	%Coordinates
	%Plane Vertex Points
	\coordinate (x1) at (3,0,0);
	\coordinate (x2) at (3,4,0);
	\coordinate (x3) at (3,4,-1.5);
        \coordinate (x4) at (3,4,2);
        \coordinate (x5) at (3,-2,-1.5);
        \coordinate (x6) at (3,-2,2);
        \coordinate (x7) at (3,0,-1.5);
        \coordinate (x8) at (3,0,2);
        \coordinate (x9) at (3,-2,0);
	
	\coordinate (O) at (0,0,0);
	
	%Axis 	
	\draw[-latex] (-1,0,0) -- (6,0,0) node[pos = 1.05] {$x_1$};
	\draw[-latex] (0,-1.5,0) -- (0,3,0) node[pos = 1.05] {$x_2$};
	\draw[-latex] (0,0,-0.5) -- (0,0,3) node[pos = 1.05] {$x_3$};
	\draw[draw=black, fill=black] (O) circle (0.5pt);
	
	
	%Plane
	\path[draw=black, fill=black!20, thick, opacity = 0.8] (x3) -- (x4) -- (x6) -- (x5) -- (x3);
	\node[shift={(-2.2,-1.9)}] at (x2) {$E$};

        \path[draw=black, fill=black!20, thick, opacity = 0.8] (x7) -- (x8);
        \path[draw=black, fill=black!20, thick, opacity = 0.8] (x2) -- (x9);
        
        \draw (x1) node[thick, cross] {} node[above left] {$S_1$};

        \draw (x1) -- (x2) node[pos=0.5, below] {$s_{12}$};
        \draw (x1) -- (x7) node[pos=0.5, right] {$s_{13}$};

        \draw (x1) -- (5.5,0,0);
	
	
    \end{tikzpicture}
\end{minipage}

\ \\
\ \\
\ \\
\ \\
\ \\
\ \\

\section{Gegenseitige Lage von Ebenen und Geraden}

Gegeben sind die Gerade $g$ mit $g: \ \vv{x} = \vv{p} + r\vv{u}$ und die Ebene $E$ in Parameterform mit $E: \ \vv{x} = \vv{q} + s\vv{v} + t\vv{w}$. $\vv{n}$ sei ein Normalenvektor von $E$. 

Es gibt drei Fälle, wie eine Gerade zu einer Ebene liegen kann:

\begin{center}
    \begin{tabular}{ | p{0.33\textwidth} | p{0.33\textwidth} | p{0.33\textwidth} | }
        \hline
         \begin{center} $g$ schneidet $E$ in einem Durchstoßpunkt  $D$. $g \cap E = \{D\}$ \end{center} & \begin{center} $g$ liegt in $E$ \qquad $g \subset E, (g \parallel E)$ \end{center}& \begin{center} $g$ ist parallel zu $E$ und liegt nicht in $E$  $g \parallel E \land g \not\subset E$, d. h. $g \cap E = \{\}$  \end{center}\\
         \hline
         \begin{center} \begin{tikzpicture}[x={(-0.7071cm,-0.7071cm)}, y={(1cm,0.0cm)}, z={(0cm,1cm)}, line cap=round, line join=round,scale = 0.6]
	%Coordinates
	%Plane Vertex Points
	\coordinate (x1) at (1,-3,-1);
	\coordinate (x2) at (1,3,-1);
	\coordinate (x3) at (-1,-3,1);
        \coordinate (x4) at (-1,3,1);

        \coordinate (x5) at (1,0,5);
        \coordinate (x6) at (-1,0,-5);

	
	\coordinate (O) at (0,0,0);
	
        \draw (x5)  node[below right] {$g$} -- (x6);
 
	%Plane
	\path[draw=black, fill=black!20, thick, opacity = 0.8] (x1) -- (x2) -- (x4) -- (x3) -- (x1);
	\node[shift={(-2.4,-1.5)}] at (x2) {$E$};

        \draw (O) node[thick, cross] {} node[right] {$D$} -- (x5);

    \end{tikzpicture} \end{center} & \vspace*{45px} \begin{center} \begin{tikzpicture}[x={(-0.7071cm,-0.7071cm)}, y={(1cm,0.0cm)}, z={(0cm,1cm)}, line cap=round, line join=round,scale = 0.6]
	%Coordinates
	%Plane Vertex Points
	\coordinate (x1) at (1,-3,-1);
	\coordinate (x2) at (1,3,-1);
	\coordinate (x3) at (-1,-3,1);
        \coordinate (x4) at (-1,3,1);

        \coordinate (x5) at (0,-2.85,0);
        \coordinate (x6) at (0,2.85,0);

	
	\coordinate (O) at (0,0,0);
 
	%Plane
	\path[draw=black, fill=black!20, thick, opacity = 0.8] (x1) -- (x2) -- (x4) -- (x3) -- (x1);
	\node[shift={(-2.4,-1.5)}] at (x2) {$E$};

        \draw (x5)  node[below right] {$g$} -- (x6);

    \end{tikzpicture} \end{center} & \vspace*{25px} \begin{center} \begin{tikzpicture}[x={(-0.7071cm,-0.7071cm)}, y={(1cm,0.0cm)}, z={(0cm,1cm)}, line cap=round, line join=round,scale = 0.6]
	%Coordinates
	%Plane Vertex Points
	\coordinate (x1) at (1,-3,-1);
	\coordinate (x2) at (1,3,-1);
	\coordinate (x3) at (-1,-3,1);
        \coordinate (x4) at (-1,3,1);

        \coordinate (x5) at (0,-2.85,3);
        \coordinate (x6) at (0,2.85,3);

	
	\coordinate (O) at (0,0,0);
 
	%Plane
	\path[draw=black, fill=black!20, thick, opacity = 0.8] (x1) -- (x2) -- (x4) -- (x3) -- (x1);
	\node[shift={(-2.4,-1.5)}] at (x2) {$E$};

        \draw (x5)  node[below right] {$g$} -- (x6);

    \end{tikzpicture} \end{center}\\
    \hline
    Das LGS \newline $\vv{p} + r\vv{u} = \vv{q} + s\vv{v} + t\vv{w}$ \newline hat eine Lösung & Das LGS \newline $\vv{p} + r\vv{u} = \vv{q} + s\vv{v} + t\vv{w}$ \newline hat unendlich viele Lösungen (wahre Aussage) & Das LGS \newline $\vv{p} + r\vv{u} = \vv{q} + s\vv{v} + t\vv{w}$ \newline hat keine Lösung (falsche Aussage) \\
    \hline
    $\vv{u}$, $\vv{v}$ und $\vv{w}$ sind \newline linear unabhängig & $\vv{u}$, $\vv{v}$ und $\vv{w}$ sind \newline linear abhängig & $\vv{u}$, $\vv{v}$ und $\vv{w}$ sind \newline linear abhängig \\
    \hline
    & $\vv{q-p}$, $\vv{v}$ und $\vv{w}$ sind \newline linear abhängig & $\vv{q-p}$, $\vv{v}$ und $\vv{w}$ sind \newline linear unabhängig \\
    \hline 
    $\vv{u}$ und $\vv{n}$ sind weder orthogonal, noch parallel, d. h. \newline $\vv{u} \circ \vv{n} \neq 0$ und $\vv{u} \neq k \cdot \vv{n}$ \footnotemark & $\vv{u}$ und $\vv{n}$ sind orthogonal zueinander. d. h. \newline $\vv{u} \circ \vv{n} = 0$ & $\vv{u}$ und $\vv{n}$ sind orthogonal zueinander. d. h. \newline $\vv{u} \circ \vv{n} = 0$ \\ 
    \hline
    \end{tabular}
\end{center}

\footnotetext{Achtung Sonderfall: orthogonaler Schnitt $\rightarrow \vv{u}$ und $\vv{n}$ sind Vielfache, d. h.: $\vv{u} = k \cdot \vv{n}$}

\ \\
\ \\

Gegeben sind die Gerade $g$ mit $g: \ \vv{x} = \vv{p} + r\vv{u}$ und die Ebene $E$ in Koordinatenform mit $E: \ ax_1 + bx_2 +cx_3 = d$ 

Es gibt wieder drei Fälle, wie eine Gerade zu einer Ebene liegen kann:

\begin{tabular}{ | p{0.33\textwidth} | p{0.33\textwidth} | p{0.33\textwidth} | }
    \hline
    \begin{center} $g \cap E = \{D\}$ \end{center} & \begin{center} $g \subset E, (g \parallel E)$ \end{center}& \begin{center} $g \parallel E \land g \not\subset E$, d. h. $g \cap E = \{\}$  \end{center}\\
    \hline
    \multicolumn{3}{| p{1\textwidth} |}{\begin{center} Setze Koordinaten von $g$ in Koordinatengleichung von $E$ ein. $a\left(p_1+ru_1\right)+b\left(p_2+ru_2\right)+c\left(p_3+ru_3\right)=d$ \end{center}} \\
    \hline
    Die Gleichung \newline hat genau eine Lösung für $r$ & Die Gleichung \newline ist allgemein gültig \newline $\Rightarrow$ wahre Aussage & Die Gleichung \newline hat keine Lösung \newline $\Rightarrow$ falsche Aussage \\
    \hline
\end{tabular}

\textit{Beispiel: Bestimmung eines Schnittpunktes bei gegebener Koordinatenform}


\begin{align*}
    E_a: & \ 2ax_1 + x_2 + ax_3 = 1 \qquad ,a\in\mathbb{R} \ \footnotemark && \\
    g: & \ \vv{x} = \left(\begin{array}{c}  1 \\ 1 \\ 3 \end{array}\right) + \lambda \left(\begin{array}{c}  0 \\ -1 \\ 1 \end{array}\right) && \\
    \ \newline \\
    g \cap E_a: & \ 2a(1 + \lambda \cdot 0) + (1 + \lambda \cdot (-1)) + a(3 + \lambda \cdot 1) = 1 && \\
    & \ 2a + 1 - \lambda + 3a + a\lambda = 1 && \\
    & \ \lambda = \frac{5a}{1-a}
\end{align*}

\footnotetext{Bei $E_a$ handelt es sich um seine sog. Ebenenschar}

Für $a \neq 1$ gibt es genau einen Schnittpunkt $S$ \\
$S\left(1 \ | \ 1 - \frac{5a}{1-a} \ | \ 3 + \frac{5a}{1-a}\right)$

\ \\
\ \\
\ \\
\ \\
\ \\ 
\ \\

\section{Gegenseitige Lage von Ebenen}

Gegeben sind zwei Ebenen $E$ und $E^*$ in Koordinatenform bzw. in Normalenform. \\
Koordinatenform: $E: \  ax_1 + bx_2 + cx_3 = d$ und $E^*: \  a^*x_1 + b^*x_2 + c^*x_3 = d^*$ \\
Normalenform: $E: \ (\vv{x} - \vv{p}) \circ \vv{n} = 0$ und $E^*: \ (\vv{x} - \vv{p}) \circ \vv{n^*} = 0$ 

\begin{tabular}{ | p{0.33\textwidth} | p{0.33\textwidth} | p{0.33\textwidth} | }
    \hline
    \begin{center} $E$ und $E^*$ schneiden sich in einer Geraden \qquad $E \cap E^* = g$ \end{center} & \begin{center} $E$ und $E^*$ sind paralell zueinander $E \parallel E^* \land E \cap E^* = \{\}$ \end{center}& \begin{center} $E$ und $E^*$ sind identisch \newline \ \newline $E = E^*$  \end{center}\\
    \hline
    \begin{center} \begin{tikzpicture}[x={(-0.7071cm,-0.7071cm)}, y={(1cm,0.0cm)}, z={(0cm,1cm)}, line cap=round, line join=round,scale = 0.6]
	\coordinate (x1) at (1,-3,-1);
	\coordinate (x2) at (1,3,-1);
	\coordinate (x3) at (-1,-3,1);
        \coordinate (x4) at (-1,3,1);

        \coordinate (x4a) at (-1,-0.1,1);
        \coordinate (x4b) at (-1,-2,1);

        \coordinate (x5) at (1,1,-3);
	\coordinate (x6) at (1,-1,1.5);
	\coordinate (x8) at (-1,-1,3);
        \coordinate (x7) at (-1,1,-1.5);

        \coordinate (x8a) at (1,0.11,-1);

        \coordinate (n1a) at (1,2.5,1);
        \coordinate (n1b) at (0.5,2.5,-0.5);

        \coordinate (n2a) at (-0.75,-1,2.25);
        \coordinate (n2b) at (-0.5,0,2.75);
	
	\coordinate (O) at (0,0,0);
 
        %drawing
        
        \draw[thick] (x4a) -- (x4b);
        
        \path[draw=black, fill=black!20, thick, opacity = 0.8] (x5) -- (x6) -- (x8) -- (x7) -- (x5);
	\node[shift={(-1,-0.65)}] at (x5) {$E^*$};
 
	\path[draw=black, fill=black!20, thick, opacity = 0.8] (x4b) -- (x3) -- (x1) -- (x2)  -- (x4) -- (x4a);
	\node[shift={(-2.4,-1.5)}] at (x2) {$E$};

        \draw[thick, dashed] (x8a) -- (x4a) node[pos=0.5, right] {$g$};

        \draw[thick] (x5) -- (x6);
        \draw[thick] (x6) -- (x8);

        \draw[-latex, thick,draw=black] (n1b) -- (n1a) node[below right] {$\vv{n}$};
        \draw[-latex, thick,draw=black] (n2a) -- (n2b) node[above] {$\vv{n^*}$};
    \end{tikzpicture} \end{center} & \vspace*{25px} \begin{center} \begin{tikzpicture}[x={(-0.7071cm,-0.7071cm)}, y={(1cm,0.0cm)}, z={(0cm,1cm)}, line cap=round, line join=round,scale = 0.6]
	\coordinate (x1) at (1,-3,-1);
	\coordinate (x2) at (1,3,-1);
	\coordinate (x3) at (-1,-3,1);
        \coordinate (x4) at (-1,3,1);

        \coordinate (x5) at (1,-3,1.5);
	\coordinate (x6) at (1,2.5,1.5);
	\coordinate (x7) at (-1,-3,3.5);
        \coordinate (x8) at (-1,2.5,3.5);
        
        \coordinate (n1a) at (1,2.5,1);
        \coordinate (n1b) at (0.5,2.5,-0.5);

        \coordinate (n2a) at (1,2,3.5);
        \coordinate (n2b) at (0.5,2,2);

	
	\coordinate (O) at (0,0,0);
 
        %drawing
        \path[draw=black, fill=black!20, thick, opacity = 0.8] (x3) -- (x1) -- (x2)  -- (x4) -- (x3);
	\node[shift={(-2.4,-1.5)}] at (x2) {$E$};
 
        \path[draw=black, fill=black!20, thick, opacity = 0.8] (x5) -- (x6) -- (x8) -- (x7) -- (x5);
	\node[shift={(-0.6,0.1)}] at (x5) {$E^*$};

        \draw[-latex, thick,draw=black] (n1b) -- (n1a) node[below right] {$\vv{n}$};
        \draw[-latex, thick,draw=black] (n2b) -- (n2a) node[right] {$\vv{n^*}$};
    \end{tikzpicture} \end{center} & \vspace*{45px} \begin{center} \begin{tikzpicture}[x={(-0.7071cm,-0.7071cm)}, y={(1cm,0.0cm)}, z={(0cm,1cm)}, line cap=round, line join=round,scale = 0.6]
	\coordinate (x1) at (1,-3,-1);
	\coordinate (x2) at (1,3,-1);
	\coordinate (x3) at (-1,-3,1);
        \coordinate (x4) at (-1,3,1);
        
        \coordinate (n1a) at (1,2.5,1);
        \coordinate (n1b) at (0.5,2.5,-0.5);

        \coordinate (n2a) at (1,-0.5,2);
        \coordinate (n2b) at (0.5,-0.5,0.5);

	\coordinate (O) at (0,0,0);
 
        %drawing
        \path[draw=black, fill=black!20, thick, opacity = 0.8] (x3) -- (x1) -- (x2)  -- (x4) -- (x3);
	\node[shift={(-2,-1.5)}] at (x2) {$E$} node[shift={(-2.4,-0.6)}] at (x1) {$E^*$};
 
        \draw[-latex, thick,draw=black] (n1b) -- (n1a) node[below right] {$\vv{n}$};
        \draw[-latex, thick,draw=black] (n2b) -- (n2a) node[right] {$\vv{n^*}$};
    \end{tikzpicture} \end{center} \\
    \hline
    \multicolumn{3}{| p{1\textwidth} |}{\begin{center} \textbf{1. mögliche Vorgehensweise} \end{center}} \\
    \hline
    Schreibe die zwei Gleichungen als LGS auf. \newline $ax_1 + bx_2 + cx_3 = d$ \newline $a^*x_1 + b^*x_2 + c^*x_3 = d^*$ \newline Die Lösungen für $x_1$, $x_2$ und $x_3$ beschreiben die \underline{Schnittgerade}.& Schreibe die zwei Gleichungen als LGS auf. \newline $ax_1 + bx_2 + cx_3 = d$ \newline $a^*x_1 + b^*x_2 + c^*x_3 = d^*$ \newline Das LGS hat keine Lösung (falsche Aussage).& Schreibe die zwei Gleichungen als LGS auf. \newline $ax_1 + bx_2 + cx_3 = d$ \newline $a^*x_1 + b^*x_2 + c^*x_3 = d^*$ \newline Das LGS hat unendlich viele Lösungen (wahre Aussage).\\
    \hline
    \multicolumn{3}{| p{1\textwidth} |}{\begin{center} \textbf{2. mögliche Vorgehensweise} \end{center}} \\
    \hline
    Die Vektoren $\vv{n}$ und $\vv{n^*}$ sind linear unabhängig, d. h. nicht parallel& Die Vektoren $\vv{n}$ und $\vv{n^*}$ sind linear abhängig, d. h. parallel \newline Punktprobe mit $P$ in $E^*$ bzw. mit $Q$ in $E$ ergibt eine falsche Aussage ($P \not\in E^*; Q \not\in E$)
    &  Die Vektoren $\vv{n}$ und $\vv{n^*}$ sind linear abhängig, d. h. parallel \newline Punktprobe mit $P$ in $E^*$ bzw. mit $Q$ in $E$ ergibt eine wahre Aussage ($P \in E^*; Q \in E$) \\
    \hline
\end{tabular}

Gegeben sind zwei Ebenen $E$ und $E^*$ in Parameterform. \\
$E: \ \vv{x} = \vv{p} + r\cdot \vv{u} + s\cdot \vv{v} \ ;r,s \in \mathbb{R}$ und $E^*: \ \vv{x} = \vv{p^*} + t\cdot \vv{u^*} + z\cdot \vv{v^*} \ ;t,z \in \mathbb{R}$

\begin{tabular}{ | p{0.33\textwidth} | p{0.33\textwidth} | p{0.33\textwidth} | }
    \hline
    \begin{center} $E$ und $E^*$ schneiden sich \end{center} & \begin{center} $E$ und $E^*$ sind zueinander parallel und haben keine gemeinsamen Punkte \end{center} & \begin{center} $E$ und $E^*$ sind identisch\end{center}\\
    \hline
    \begin{center} 
    \def\t{-0.9}
    \def\s{0.9}
    \def\tt{0.5}
    \def\kk{0.5}
    \def\ttt{-0.76}
    \def\sss{0.1}
    \def\tttt{-0.1}
    \def\ssss{0.1}
    \def\ttttt{-0.24}
    \def\sssss{0.9}
    \begin{tikzpicture}[x={(-0.7071cm,-0.7071cm)}, y={(1cm,0.0cm)}, z={(0cm,1cm)}, line cap=round, line join=round,scale = 0.6]
	\coordinate (x1) at (1,-3,-1);
	\coordinate (x2) at (1,3,-1);
	\coordinate (x3) at (-1,-3,1);
        \coordinate (x4) at (-1,3,1);
        \coordinate (x4a) at (-1,-0.1,1);
        \coordinate (x4b) at (-1,-2,1);
        \coordinate (x5) at (1,1,-3);
	\coordinate (x6) at (1,-1,1.5);
	\coordinate (x8) at (-1,-1,3);
        \coordinate (x7) at (-1,1,-1.5);
        \coordinate (x8a) at (1,0.11,-1);

        %Vectors Parallel to Plane
	\coordinate (n1) at ($(x4) - (x2)$);
	\coordinate (n2) at ($(x4) - (x3)$); 
        \coordinate (n3) at ($(x7) - (x5)$);
	\coordinate (n4) at ($(x7) - (x8)$); 
 
	%Points on Plane
	\coordinate (x9) at ($(x1) + \s*(n1) - \t*(n2)$);
	\node[outer sep = -0.5pt, inner sep = -0.5pt] (x10) at ($(x2) + \kk*(n1) - \tt*(n2)$) {};
	\coordinate (x11) at ($(x1) + \sss*(n1) - \ttt*(n2)$);
        \coordinate (x12) at ($(x6) + \ssss*(n3) - \tttt*(n4)$);
	\node[outer sep = -0.5pt, inner sep = -0.5pt] (x13) at ($(x5) + \kk*(n3) - \tt*(n4)$) {};
	\coordinate (x14) at ($(x6) + \sssss*(n3) - \ttttt*(n4)$);
	\coordinate (O) at (0,0,0);
 
        %drawing
        \draw[thick] (x4a) -- (x4b);
        \path[draw=black, fill=black!20, thick, opacity = 0.8] (x5) -- (x6) -- (x8) -- (x7) -- (x5);
	\node[shift={(-1,-0.65)}] at (x5) {$E^*$};
	\path[draw=black, fill=black!20, thick, opacity = 0.8] (x4b) -- (x3) -- (x1) -- (x2)  -- (x4) -- (x4a);
	\node[above right] at (x1) {$E$};
        \draw[thick] (x8a) -- (x4a);
        \draw[thick] (x5) -- (x6);
        \draw[thick] (x6) -- (x8);
        \draw[-latex, thick,draw=blue] (x10) -- (x9) node[blue, pos=0.6, above] {$\vb{\vv{v}}$};
        \draw[-latex, thick,draw=red] (x10) -- (x11) node[red, pos=0.6, above] {$\vb{\vv{u}}$};
        \draw[-latex, thick,draw=red] (x13) -- (x12) node[red, pos=0.6, below] {$\vb{\vv{u^*}}$};
        \draw[-latex, thick,draw=blue] (x13) -- (x14) node[blue, pos=0.6, left] {$\vb{\vv{v^*}}$};
    \end{tikzpicture} \end{center} & 
    \begin{center} \def\t{0.8}
    \def\s{0.2}
    \def\ss{0.6}
    \def\sss{0.8}
    \def\ttt{0.6}
    \def\ssss{0.1}
    \def\tttt{0.5}
    \def\ttttt{0.4}
    \begin{tikzpicture}[x={(-0.7071cm,-0.7071cm)}, y={(1cm,0.0cm)}, z={(0cm,1cm)}, line cap=round, line join=round,scale = 0.6]
	\coordinate (x1) at (1,1,1);
	\coordinate (x2) at (1,5,2);
        \coordinate (x3) at (-1,5,2);
	\coordinate (x4) at (-1,1,1);
        \coordinate (x5) at (1,1,3);
	\coordinate (x6) at (1,5,4);
        \coordinate (x7) at (-1,5,4);
	\coordinate (x8) at (-1,1,3);
        \coordinate (O) at (0,3,-3);

        %Vectors Parallel to Plane
	\coordinate (n1) at ($(x3) - (x2)$);
	\coordinate (n2) at ($(x3) - (x4)$); 
        \coordinate (n3) at ($(x7) - (x6)$);
	\coordinate (n4) at ($(x7) - (x8)$); 
 
	%Points on Plane
	\node[outer sep = -0.5pt, inner sep = -0.5pt] (x9) at ($(x2) + \s*(n1) - \t*(n2)$) {};
	\node[outer sep = -0.5pt, inner sep = -0.5pt] (x10) at ($(x6) + \ss*(n3) - \t*(n4)$) {};
        \coordinate (x11) at ($(x2) + \sss*(n1) - \ttt*(n2)$);
        \coordinate (x12) at ($(x2) + \ssss*(n1) - \tttt*(n2)$);
        \coordinate (x13) at ($(x6) + \sss*(n3) - \ttttt*(n4)$);
        \coordinate (x14) at ($(x6) + \ssss*(n3) - \ttttt*(n4)$);
 
        %drawing
        \draw[-latex, thick,draw=magenta] (O) node[right] {$O$} -- (x9) node[magenta, pos=0.2, left] {$\vv{p}$};
        \draw[-latex, thick,draw=magenta] (O) -- (x10) node[magenta, pos=0.2, right] {$\vv{p^*}$}; 
        \path[draw=black, fill=black!20, thick, opacity = 0.8] (x1) -- (x2) -- (x3) -- (x4) -- (x1);
	\node[right] at (x3) {$E$};
        \draw[-latex, thick,draw=magenta] (x9) -- (x10) node[magenta, pos = 0.5, left] {$\vv{p^*} - \vv{p}$};
        \path[draw=black, fill=black!20, thick, opacity = 0.8] (x5) -- (x6) -- (x7) -- (x8) -- (x5);
	\node[right] at (x7) {$E^*$};
        \draw (x10) circle (1pt) node[magenta, left] {$P^*$};
        \draw (x9) circle (1pt) node[magenta, left] {$P$};
        \draw[-latex, thick,draw=red] (x9) -- (x11) node[red, right] {$\vv{u}$};
        \draw[-latex, thick,draw=blue] (x9) -- (x12) node[blue, shift={(-0.25,0)}] {$\vv{v}$};
        \draw[-latex, thick,draw=red] (x10) -- (x13) node[red, right] {$\vv{u^*}$};
        \draw[-latex, thick,draw=blue] (x10) -- (x14) node[blue, shift={(-0.25,0)}] {$\vv{v^*}$};
        \draw (O) circle (1pt);
    \end{tikzpicture} \end{center} & \begin{center} \def\t{0.8}
    \def\s{0.2}
    \def\ss{0.8}
    \def\sss{0.5}
    \def\ttt{0.6}
    \def\ssss{0.1}
    \def\tttt{0.5}
    \def\sssss{0.5}
    \def\ttttt{0.15}
    \def\ssssss{0.95}
    \def\tttttt{0.4}
    \begin{tikzpicture}[x={(-0.7071cm,-0.7071cm)}, y={(1cm,0.0cm)}, z={(0cm,1cm)}, line cap=round, line join=round,scale = 0.6]
        \coordinate (x1) at (1.5,1,1);
        \coordinate (x2) at (1.5,6,2);
        \coordinate (x3) at (-2,6,2);
        \coordinate (x4) at (-2,1,1);
        \coordinate (O) at (0,3,-3);
    
        %Vectors Parallel to Plane
        \coordinate (n1) at ($(x3) - (x2)$);
        \coordinate (n2) at ($(x3) - (x4)$); 
     
        %Points on Plane
        \node[outer sep = -0.5pt, inner sep = -0.5pt] (x9) at ($(x2) + \s*(n1) - \t*(n2)$) {};
        \node[outer sep = -0.5pt, inner sep = -0.5pt] (x10) at ($(x2) + \ss*(n1) - \tt*(n2)$) {};
        \coordinate (x11) at ($(x2) + \sss*(n1) - \ttt*(n2)$);
        \coordinate (x12) at ($(x2) + \ssss*(n1) - \tttt*(n2)$);
        \coordinate (x13) at ($(x2) + \sssss*(n1) - \ttttt*(n2)$);
        \coordinate (x14) at ($(x2) + \ssssss*(n1) - \tttttt*(n2)$);
    	
        %drawing
        \draw[-latex, thick,draw=magenta] (O) node[right] {$O$} -- (x9) node[magenta, pos=0.2, left] {$\vv{p}$};
        \draw[-latex, thick,draw=magenta] (O) node[right] {$O$} -- (x10) node[magenta, pos=0.2, right] {$\vv{p^*}$};
        \path[draw=black, fill=black!20, thick, opacity = 0.8] (x1) -- (x2) -- (x3) -- (x4) -- (x1);
        \node[shift={(0.45,-0.3)}] at (x3) {$E$};
        \node[above] at (x2) {$E^*$};
        \draw (x9) circle (1pt) node[magenta, left] {$P$};
        \draw (x10) circle (1pt) node[magenta, left] {$P^*$};
        \draw[-latex, thick,draw=red] (x9) -- (x11) node[red, right] {$\vv{u}$};
        \draw[-latex, thick,draw=blue] (x9) -- (x12) node[blue, shift={(-0.25,0)}] {$\vv{v}$};
        \draw[-latex, thick,draw=red] (x10) -- (x14) node[red, above] {$\vv{u^*}$};
        \draw[-latex, thick,draw=blue] (x10) -- (x13) node[blue, shift={(-0.25,0)}] {$\vv{v^*}$};
        \draw[-latex, thick, magenta] (x9) -- (x10) node[magenta, pos = 0.6, left] {$\vv{p^*} - \vv{p}$};
        \draw (O) circle (1pt);
    \end{tikzpicture}\end{center}\\
    \hline
    $\vv{u}$, $\vv{v}$, $\vv{u^*}$  \textbf{oder} $\vv{u}$, $\vv{v}$, $\vv{v^*}$ \newline sind linear unabhängig & \multicolumn{2}{c|}{$\vv{u}$, $\vv{v}$, $\vv{u^*}$  \textbf{und} $\vv{u}$, $\vv{v}$, $\vv{v^*}$ sind linear abhängig}\\
    \hline
    & $\vv{p} - \vv{p^*}$, ?, $\vv{v}$ \newline sind linear unabhängig & $\vv{p} - \vv{p^*}$, ?, $\vv{v}$ \newline sind linear abhängig \\
    \hline
    \multicolumn{3}{| p{1\textwidth} |}{\begin{center} \textbf{3. mögliche Vorgehensweise} \end{center}} \\
    \hline
    Die Gleichung \newline $\vv{p} + r \vv{u} + s \vv{v} = \vv{p^*} + t \vv{u^*} + z \vv{v^*}$ \newline hat unendlich viele Lösungen (ist noch abhängig von einem Parameter). & Die Gleichung \newline $\vv{p} + r \vv{u} + s \vv{v} = \vv{p^*} + t \vv{u^*} + z \vv{v^*}$ \newline hat keine Lösung. LGS ergiebt eine flasche Aussage. &  Die Gleichung \newline $\vv{p} + r \vv{u} + s \vv{v} = \vv{p^*} + t \vv{u^*} + z \vv{v^*}$ hat unendlich viele Lösungen. LGS ergibt eine wahre Aussage. \\
    \hline
\end{tabular}

Um die Lagebeziehung zweier Ebenen zu untersuchen, sind die Normalengleichung oder die Koordinatengleichung am geschicktesten.

Hat man hingegen die Parametergleichung gegeben, so wandeln wir diese zuerst in eine der beiden anderen Darstellungsformen um. Zuerst untersuchen wir, ob die Normalenvektoren Vielfache sind und führen dann gegebenenfalls noch eine Punktprobe durch, um zu entscheiden, ob die Ebenen außer zueinander parallel zudem identisch sind.

Um die Schnittgerade zweier Ebenen zu bestimmen, muss ein LGS gelöst werden.

\textit{Beispiel:} Gegeben ist die Schar der Ebenen $E_a: \ 2ax_1 + x_2 + ax_3 = 1 \ , a \in \mathbb{R}$. Zeigen Sie, dass es eine Gerade $h$ gibt, die in allen Ebenen der Schar liegt.

\begin{align}
    E_0 \cap E_1 = \{h\} \tag{1} 
\end{align}
\begin{align*}
    &E_0: \ x_2 && = 1 & \\
    &E_1: \ 2x_1  +  x_2  +   x_3  && = 1 & \\
    \ \\
    & E_0 \cap E_1 : x_2 && = 1 \\   
    &2x_1 + 1  +x_3 && = 1  \\
    &2x_1 +x_3 && = 0 
\end{align*}

Setze: $x_1 = t \ ,t \in \mathbb{R}$

Damit: $x_2 = 1$ und $x_3 = -2t$

$g: \ \vv{x} = \left(\begin{array}{c}  0 \\1 \\ 0 \end{array}\right) + t \left(\begin{array}{c}  1 \\ 0 \\ -2 \end{array}\right) \, t \in \mathbb{R}$

\begin{align}
    g \text{ in } E_a \tag{2}
\end{align}
\begin{align*}
    & 2a\cdot t + 1 + a\cdot (-2t) && = 1 &\\
    & 2at + 1 -2at && = 1 &\\
    & 1 && = 1 &
\end{align*}

$\Rightarrow$ wahre Aussage $\Rightarrow$ $g$ liegt in allen Ebenen der Schar
\chapter{Exponential- und Logartihmusfunktionen}

\subsection{Wiederholung: Exponentialfunktionen}

\begin{definition}
    Funktionen mit $f$ bzw. $g$ mit $$f(x) = a^x \text{ bzw. }g(x) = c \cdot a^x \text{  wobei } a, c \in \mathbb{R} \text{ mit } a > 0 \text{ und } a,c \neq 0$$ nennt man Exponentialfunktionen zur Basis $a$.
\end{definition}

\textit{Beispiele:}\footnote{Beachte beim Zeichnen, dass die Funktionen \textbf{nicht} die $x$-Achse schneiden. Am besten kurz davor mit dem Zeichnen aufhören.}

\begin{minipage}[b]{0.2\textwidth}
    (1): $\left(\frac{1}{2}\right)^x$ \\
    (2): $3 \cdot 2^x$ \\
    (3): $2 \cdot 2^x$ \\
    (4): $2^x$ \\
    (5): $-\left(\frac{1}{2}\right)^x$ \\
    (6): $-2 \cdot 2^x$ \\
    (7): $-2^x$
\end{minipage}
\begin{minipage}{0.8\textwidth}
    \begin{tikzpicture}[scale=1.5]
    \begin{axis}[clip=true, 
        xmin=-7.5, xmax=7.5, ymin=-7.5, ymax=7.5,
        axis lines = middle, 
        xlabel=$x$, ylabel=$y$,
        xlabel style = {font=\tiny,xshift=0.5ex},
        ylabel style = {font=\tiny,yshift=0.5ex},
        xtick={-7,-6,...,6,7}, xticklabels={-7,-6,...,6,7}, yticklabel style = {font=\tiny,xshift=0.5ex},
        xticklabel style = {font=\tiny,yshift=0.5ex},
        ytick={-7,-6,...,6,7}, yticklabels={-7,-6,...,6,7}]
    \addplot[samples=50, domain=-3:7]{(1/2)^x} node[xshift=5pt, pos=0.1]{\tiny(1)};
    \addplot[samples=50, domain=-7:1.5]{3*2^x} node[xshift=-5pt, pos=0.85]{\tiny(2)};
    \addplot[samples=50, domain=-7:1.98]{2*2^x} node[xshift=5pt, pos=0.9]{\tiny(3)};
    \addplot[samples=50, domain=-7:3]{2^x} node[xshift=5pt, pos=0.9]{\tiny(4)};
    \addplot[samples=50, domain=-3:7]{-(1/2)^x} node[xshift=5pt, pos=0.1]{\tiny(5)};
    \addplot[samples=50, domain=-7:1.98]{-2*2^x} node[xshift=-5pt, pos=0.9]{\tiny(6)};
    \addplot[samples=50, domain=-7:3]{-2^x} node[xshift=5pt, pos=0.9]{\tiny(7)};
    \end{axis}
    \end{tikzpicture}
\end{minipage} 

\textbf{Eigenschaften der Exponentailfunktionen zur Basis a}\\
Eigenschaften der Graphen $f$ mit $f(x) = a^x, a>0, a\neq1$:
\begin{outline}
    \1 sie verlaufen oberhalb der $x$-Achse, d. h. die Funktionswerte von $f$ sind alle positiv (>0) 
    \1 sie gehen durch den Punkt $P(0 \ | \ 1)$, denn es gilt $a^0 = 1$
    \1 sie sind für
    \2 $a>1$ streng monoton wachsend
    \2 $a<1$ streng monoton fallend
    \1 die $x$-Achse ist \textbf{waagrechte Asymptote}, d.h. die Werte von $f$ näheren sich der $x$-Achse, entweder für $x \to \infty$ oder $x \to -\infty$
\end{outline}
Die Graphen der Funktion $g$ mit $g(x) = c\cdot a^x$ gehen aus denen von $f$ mit $f(x)=a^x$ durch Streckung mit denm Faktor $c$ in $y$-Richtung hervor.

Falls $c<0$, erfolgt eine Spiegelung an der $x$-Achse.

Das große Anwendungsgebiet der Exponentialfunktionen ist die Beschreibung von Wachstumsvorgängen. Dabei werden auch Zerfallsvorgänge mit ihrer Hilfe beschrieben.

Die Basis $a$ im Funktionsterm der allgemeinen Exponentialfunktionen heißt auch Wachstumsfaktor.

\section{Die natürliche Exponentialfunktion und die Euler'sche Zahl e}

\begin{satz}
    Für die Funktion $f$ mit $f(x) = a^x$ gilt $f^\prime(x) = f^\prime(0) \cdot a^x$. Wenn $f^\prime(0) = 1$ ist, dann gilt $$f^\prime(x) = a^x \text{ ,also } f(x) = f^\prime(x)$$
\end{satz}

\begin{satz}
    Fie Funktion $f(x) = e^x$ ist diejenige Funktion, dür die gilt: $f(x) = f^\prime(x) = e^x$. Sie heißt natürliche Exponentialfunktion. $e$ ist die Euler'sche Zahl ($e \approx 2,718$).
\end{satz}

\ \\
\ \\

\begin{minipage}{0.5\textwidth}
    \textit{Beispiele:}
    \begin{equation*}
        \begin{gathered}
            f(x) = 2 + e^x \\
            f^\prime(x) = e^x\\
            \ \\
            f(x) = -5e^x \\
            f^\prime(x) = -5e^x \\
            \ \\
            f(x) = e^{5x} \\
            f^\prime(x) = 5 e^{5x} \\
            \ \\
            f(x) = e^{3x +2} \\
             f^\prime(x) = 3e^{3x + 2}
        \end{gathered}
    \end{equation*}
\end{minipage}
\begin{minipage}{0.5\textwidth}
    Graph der natürlichen Exponentialfunktion \\ $f(x) = e^x$
    
    \begin{tikzpicture}[scale=1]
    \begin{axis}[clip=true, 
        xmin=-2.5, xmax=3.5, ymin=-0.5, ymax=5.5,
        axis lines = middle, 
        xlabel=$x$, ylabel=$y$,
        xlabel style = {font=\tiny,xshift=0.5ex},
        ylabel style = {font=\tiny,yshift=0.5ex},
        xtick={-7,-6,...,6,7}, xticklabels={-7,-6,...,6,7}, yticklabel style = {font=\tiny,xshift=0.5ex},
        xticklabel style = {font=\tiny,yshift=0.5ex},
        ytick={-7,-6,...,6,7}, yticklabels={-7,-6,...,6,7}, grid = major]
    \addplot[thick, samples=50, domain=-2.5:1.6]{(e^x} node[right, pos=0.8]{\tiny$G_f$};
    \end{axis}
    \end{tikzpicture}
\end{minipage}

\section{Exponentialgleichung und der natürliche Logarithmus}

\textit{Wiederholung:}

\begin{minipage}{0.5\textwidth}
    (1) $4^x = 12 \Rightarrow x = \log_{4}(12)$ \\
    (2) $a^x = y \Rightarrow x = \log_{a}(y)$
\end{minipage}
\begin{minipage}{0.5\textwidth}
    (3) $e^x = 12 \Rightarrow x = \log_{e}(12)$ \\
    (4) $e^x = y \Rightarrow x = \log_{e}(y)$ 
\end{minipage}

\begin{definition}
    $\log_{e}(y) =\ln{(y)}$\footnote{vor allem bei einfachen Ausdrücken auch manchmal ohne Klammern: $\ln{y}$} heißt \textbf{natürlicher Logarithmus} von $y$ und ist die Lösung der Gleichung $e^x = y$. \\Beachte: $y>0$, da $e^x >0$
\end{definition}
\noindent\rule{\textwidth}{1pt}

\textit{Beispiele:}

(1) $\ln{2} \approx 0,69$ (da $e^{0,69} \approx 2$) \\
(2) $\ln{0,25} \approx -1,39$ \\
(3) $\ln{(-3)}:$ nicht definiert

\textit{\textbf{Merke:}}

\begin{minipage}{0.5\textwidth}
    (1) $\ln{1} = 0$ (da $e^0 = 1$) \\
    (2) $\ln{e} = 1$ (da $e^1 = 1$) \\
    (3) $\ln{e^2} = 2$ (oder: $\ln{e^2} = 2 \cdot \ln{e} = 2$
\end{minipage}
\begin{minipage}{0.5\textwidth}
    (4) $\ln{e^x} = x$ \\
    (5) $\ln{\frac{1}{e}} = -1$ ($=\ln{e^{-1}})$ \\
    (6) $e^{\ln{e}} = x$
\end{minipage}


\subsection{Einschub: Ableitung der Exponentialfunktion}
Ges.: $f^\prime(x)$ für $f(x) = a^x$

Bsp.: $f(x) = 2^x = (e^{\ln{2}})^x = e^{\ln{2} \cdot x}$

\qquad \ $f^\prime(x) = \ln2 \cdot e^{\ln{2} \cdot x} = \ln{2} \cdot 2^x$

allgemein: 

$f(x) = a^x = e^{\ln{a} \cdot x}$\\
$f^\prime(x) = \ln{a} \cdot e^{\ln{a} \cdot x} = \ln{a} \cdot a^x$

\noindent\rule{\textwidth}{1pt}

\textit{Beispiele:}

\begin{minipage}[t]{0.5\textwidth}
    \begin{equation*}
        \begin{aligned}
            \text{(1.1) } && e^x &= 15 \\
            && x &= \ln{15} \approx 2,708 \\
            \ \\
            \text{(1.3) } && 3 \cdot e^{4y + 1} &= 16,2 \\
            && e^{4y + 1} &= 5,4 \\
            && 4y +1 &= \ln{5,4} \\
            && y &= \frac{1}{4}(\ln{(5,4)}-1) \approx 0,172 \\
            \ \\
            \text{(2.1) } && e^x &= e \\
            && x &= 1 \\
            \ \\
            \ \\
            \text{(2.3) } && 2 \cdot e^{7x -3} &= \sqrt{4e} \\
            && 2 \cdot e^{7x -3} &= 2\sqrt{e} \\
            && e^{7x -3} &= \sqrt{e} \\
            && 7x &= \frac{7}{2} \\
            && x &= \frac{1}{2}
        \end{aligned}
    \end{equation*}
\end{minipage}
\begin{minipage}[t]{0.5\textwidth}
    \begin{equation*}
        \begin{aligned}
            \text{(1.2) } && 4\cdot e^{-x} &= 10 \\
            && x &= - \ln{2,5} \approx -0,916 \\
            \ \\
            \text{(1.4) } && 3 + 2\cdot e^{3-4a} &= 9 \\
            && 4- 4a &= \ln{3} \\
            && a &= \frac{1}{4}(3 - \ln{3}) \approx0,475 \\
            \ \\
            \ \\
            \text{(2.2) } && e^{x -1} &= \sqrt{e} \\
            && x &= \frac{3}{2} \\
            \ \\
            \text{(2.4) } && e^{-4z} &= \frac{1}{e^2} \\
            && z &= \frac{1}{2}
        \end{aligned}
    \end{equation*}
\end{minipage}

\ \\
\ \\
\ \\

\subsection{Lösen von Exponentialgleichungen}

(1) Strategie: $e^x$ ausklammern
\begin{equation*}
    \begin{aligned}
        && e^{2x} - 3e^x & = 0 && \\
        && e^x(e^x -3) & = 0 && \\
        \ \\
        \text{S. v. Np.: } \text{ (1) } && e^x & \neq 0 \text{ für alle } x \in \mathbb{R} && \\
        \text{ (2) } && e^x - 3 & = 0 &&\\
        && x & = \ln{3} &&
    \end{aligned}
\end{equation*}

(2) Strategie: Multiplizieren mit $e^x$
\begin{equation*}
    \begin{aligned}
        e^x - 3e^{-x} & = 0 \\
        e^x - \frac{3}{e^x} & = 0 \qquad | \cdot e^x \\
        e^{2x} - 3 & = 0 \\
        x & = \frac{1}{2} \ln{3}
    \end{aligned}
\end{equation*}


(3) Strategie: Substitution $u = e^x$

{\centering $e^{2x} - 3e^x + 2 = 0$\par }

\begin{minipage}{0.5\textwidth}
    \begin{align*}
    \text{Substitution: } u & = e^x \\
    u^2 - 3u + 2 & = 0 \\
    \text{Mitternachtsformel:} & \\
    u_{1,2} & = \frac{3 \pm \sqrt{9 - 8}}{2} \\
     &= \frac{3 \pm 1}{2} \\
    \Rightarrow u_1 = 2 \quad u_2 & = 1 \\
\end{align*}
\end{minipage}
\begin{minipage}{0.5\textwidth}
    \begin{align*}
    \text{Resubstitution:} \\
    \text{(1) } e^x & = 2 \\
    x_1 & = \ln{2} \\
    \text{(2) } e^x & = 1 \\
    x_2 & = \ln{1} = 0
\end{align*}
\end{minipage}

(4)
\begin{align*}
    e^x - 2 e^{-x} +1 & = 0 \\
    e^x - \frac{2}{e^x} +1 & = 0  && | \cdot e^x\\
    e^{2x} -2 + e^x & = 0  && | \text{ Substitution: } u = e^x \\
    u^2 + u -2 & = 0 \\
\end{align*} 

(5)
\begin{align*}
    \left(e^x - \frac{1}{2}\right) \left(1 - e^x\right) = 0 \\
\end{align*} 
S. v. Np.:

\begin{minipage}[t]{0.5\textwidth}
    \begin{align*}
        \text{(1) }e^x - \frac{1}{2} & = 0 \\
        x & = \ln{\frac{1}{2}} = \ln{1} - \ln{2} \\
        x_1 & = - \ln{2}
    \end{align*}
\end{minipage}
\begin{minipage}[t]{0.5\textwidth}
    \begin{align*}
        \text{(2) }1 - e^x & = 0\\
        x_2 & = \ln{1} = 0
    \end{align*}
\end{minipage}

\section{Graphen von Exponentialfunktionen}
\subsection{Verhalten für $x \to \pm \infty$}
\textit{Beispiel:}

$f(x) = x^2e^{2x}$

Für $x \to \infty$ gilt: $f(x) \to \infty$ \\
Für $x \to - \infty$ gilt: $f(x) \to 0$

Der Graph von $f$ nähert sich für $x \to \infty$ der $x$-Achse, die Gerade $y = 0$ ist eine \textbf{waagrechte Asymptote} des Graphen von $f$ für $x \to \infty$.

\noindent\rule{\textwidth}{1pt}

\textit{Weitere Beispiele für Verhalten von $x \to \pm \infty$ bei Exponentialfunktionen:}

\begin{minipage}{0.43\textwidth}
    Für $x \to +\infty$ gilt \quad $e^x \to +\infty$ \\
    Für $x \to -\infty$ gilt \quad $e^x \to 0$ \\
    \ \\
    Für $x \to +\infty$ gilt \quad $e^{-x} \to 0$ \\
    Für $x \to -\infty$ gilt \quad $e^{-x} \to +\infty$ \\
    \ \\
    \textbf{Merke:} Für große Werte von $x$ \textbf{wächst} $e^x$ stets \textbf{\glqq schneller\grqq} als $x^n$\\
    \ \\
    Für $x \to +\infty$ gilt \quad $\frac{e^x}{x^n} \to +\infty$ \\
    Für $x \to -\infty$ gilt \quad $\frac{e^x}{x^n} \to 0$ \\
\end{minipage}
\begin{minipage}[t]{0.01\textwidth}
    \
\end{minipage}
\begin{minipage}{0.56\textwidth}
    Für $x \to +\infty$ gilt \quad $\frac{x^n}{e^x} \to 0$ \\
    Für $x \to -\infty$ gilt \quad $\frac{x^n}{e^x} \to +\infty$ (n gerade)\\
    $\land$\qquad \qquad \qquad \quad \ \ \qquad $\to -\infty$ (n ungerade) \\
    \ \\
    Für $x \to +\infty$ gilt \quad $x^n \cdot e^x \to +\infty$ \\
    Für $x \to -\infty$ gilt \quad $x^n \cdot e^x \to 0$ \\
    \ \\
    Für $x \to +\infty$ gilt \quad $e^x - x^n \to +\infty$ \\
    Für $x \to -\infty$ gilt \quad $e^x - x^n \to -\infty$ (n gerade) \\
    $\land$\qquad \qquad \qquad \qquad \quad \, \ \qquad $\to +\infty$ (n ungerade) \\
    \ \\
\end{minipage}

\subsection{Untersuchen auf Symmetrie}
\begin{satz}
    Ein Graph $f$ ist \textbf{achsensymmetrisch} zur $y$-Achse wenn $f(-x) = f(x)$ gilt. \\
    \phantom \qquad \ \ \ \ \ \, Ein Graph $f$ ist \textbf{punktsymmetrisch} zum Ursprung wenn $f(-x) = -f(x)$ gilt. 
\end{satz}

\textit{Beispiele:} \\
(1)
\begin{align*}
    f(x) & = 5xe^{-x^2} && \\
    f(-x) & = 5(-x)e^{-(-x^2)} = -5xe^{-x^2} = -f(x) &&
\end{align*}
$\rightarrow$ Der Graph von $f$ verläuft punktsymmetrisch zum Ursprung, da $f(-x) = -f(x)$

(2)
\begin{align*}
    f(x) & = e^{x^2}\left(e^{-2x} + e^{2x}\right) && \\
    f(-x) & = e^{x^2}\left(e^{2x} + e^{-2x}\right) &&
\end{align*}
$\rightarrow$ Der Graph von $f$ ist achsenysmmetrisch zur $y$-Achse, da $f(-x) = f(x)$

\section{Exponentailfunktionen mit Parameter}

\begin{definition}
    Enthält eine Funktion $f$ neben der unabhängigen Variablen $x$ noch einen \textbf{Parameter}, z. B. $t$, so gehört zu jedem $t$ eine Funktion $f_t$, die jedem $x$-Wert den Funktionswert $f_t(x)$ zuordnet. Die Funktionen $f_t$ bilden eine \textbf{Funktionenschar} von Funktionen.
\end{definition}
\textit{Beispiel:} \\
$f_t(x) = e^{2t - x} + x - 3t$\\
$f^\prime_t(x) = -e^{2t-x} + 1$\\
$f^{\prime\prime}_t(x) = e^{2t - x}$

Wir können Funktionsscharen wie Funktionen auf charakteristische Eigenschaften untersuchen, z. B. Symmetrie, Nullstellen, Extremstellen, Verhalten für $|x| \to \infty$, Asymptoten 

\textit{Beispiel:}\\
$f^\prime_t(x) = 0 \Longleftrightarrow 0 = -e^{2t - x} \Longleftrightarrow x = 2t$\\
$f^{\prime\prime}_t(x) = e^0 = 1 \Rightarrow \text{ Minimumstelle}$ \\
Also: $T_t( 2t \ | \ 1-t )$

Um ein Schaubild $K_f$ einer Funktion $f_t$ zu zeichnen, berechnen wir noch die fehlenden $y$-Werte der Punkte bzw. auch noch einen Schnittpunkt mit der $y$- Achse. 

\textit{Beispiel:}\\
$f_t(0) = e^{2t} -3t \Rightarrow S_y\left(0 \ | \ e^{2t} -3t \ \right)$
\begin{definition}
    \textbf{Ortskurve:} Ist nicht nur eine Funktion $f$, sondern eine Funktionenschar $f_t$ mit dem Parameter $t$ gegeben, sind häufig die Koordinaten bestimmter Punkte $P_t$ (z. B. Extrem- oder Wendepunkte) von $t$ abhängig. Durchläuft $t$ alle zugelassenen Werte, so bilden die Punkte $P_t$ eine Kurve. Diese Kurve heißt \textbf{Ortslinie}, \textbf{Ortskurve} oder \textbf{geometrischer Punkt} der Punkte $P_t$. \\
    Die zugehörigen Gleichung ergibt sich, indem wir $t$ aus den beiden Gleichungen $x = x(t)$ und $y = y(t)$ eliminieren.
\end{definition}
\textit{Beispiel:} \\
$T_t(2t \ | \ 1-t )$ \\
$x = 2t \Longleftrightarrow t = \frac{x}{2} \ (*)$\\
$(*) \text{ in } y = 1 - t: y = 1 - \frac{x}{2}$

Manche Funktionsscharen besitzen gemeinsame Punkte, d. h. es gibt Punkte, die auf allen zugehörigen Schaubildern liegen, wenn $t$ alle zugelassenen Werte durchläuft. Sie sind dann unabhängig von $t$. 

\textit{Beispiel:}\\
$f_a(x) = f_b(x) \qquad , a \neq b$\\
Die gesuchten Punkte sind unabhängig vom Parameter.

\noindent\rule{\textwidth}{1pt}

\textit{weiteres Beispiel:} \\
Gegeben ist die Funktionenschar $f_a$ mit $f_a(x) = e^x - a$ \ ($a \in \mathbb{R}$)\\
a) Geben sie $f_3(x)$, $f_{-2}(x)$ und $f_0(x)$ an.\\
b) Beschreiben Sie, wie sich eine Erhöhung von $a$ auf die Lage der Graphen auswirkt.\\
c) Bestimmen Sie die Ableitung von $f_a$

a) $f_3(x) = e^x -3$, $f_{-2}(x) = e^x +2$, $f_0(x) = e^x$ \\
b) Eine Erhöhung von $a$ bewirkt eine Verschiebung des Graphen nach unten. \\
c) $f^\prime_a(x) = e^x$
\ \\
\ \\
\ \\
\ \\
\ \\
\ \\
\ \\
\ \\
\section{Die Umkehrfunktion}

\begin{definition}
    Eine Funktion $f$ mit der Definitonsmenge $D_f$ und Wertemenge $W_f$ heißt \textbf{umkehrbar}, falls es zu jedem $y \in W_f$ genau ein $x \in D_f$ mit $f(x) = y$ gibt. \\
    Bei einer umkehrbaren Funktion $f$ heißt die Funktion $\bar{f}$ mit $\bar{f}(y) = x$ (wobei $f(x) = y$ ist) die \textbf{Umkehrfunktion} von $f$. Sie hat die Defintionsmenge $D_{\bar{f}} = W_f$ und die Wertemenge $W_{\bar{f}} = D_f$.
    $$\text{Es gilt } \bar{f}(f(x)) = x \text{ für alle } x \in D_f \text{ und } f(\bar{f}(x)) = x \text{ für alle } x \in D_{\bar{f}} = W_f$$
\end{definition}

\begin{minipage}{0.35\textwidth}
    \textit{Beispiel} \\
    $f(x) = x^2$ \quad mit $D_f = \mathbb{R}^+_0$ \\
    
    $\Rightarrow$ Der Graph von $\bar{f}$ entsteht aus dem Graphen von $f$ durch Spiegelung an der 1. Winkelhalbierenden.
    \begin{align*}
        \Rightarrow f(x) & = x^2 \\
        y & = x ^2 \\
        x & = \sqrt{y} \text{ oder } x = -\sqrt{y}
    \end{align*}
    Da $x \in \mathbb{R}^+_0$, trifft der 1. Fall zu \\
    Also: $y= \sqrt{x}$ \\
    $\bar{f}(x) = \sqrt{x}$
    
\end{minipage}
\begin{minipage}{0.65\textwidth}
    \begin{tikzpicture}[scale=1.5]
        \begin{axis}[clip=false, 
            xmin=-2.5, xmax=4.5, ymin=-1.5, ymax=4.5,
            axis lines = middle, 
            xlabel=$x$, ylabel=$y$,
            xlabel style = {font=\tiny,xshift=0.5ex},
            ylabel style = {font=\tiny,yshift=0.5ex},
            xtick={-2,-1,...,3,4}, xticklabels={-2,-1,...,3,4}, yticklabel style = {font=\tiny,xshift=0.5ex},
            xticklabel style = {font=\tiny,yshift=0.5ex},
            ytick={-1,0,...,3,4}, yticklabels={-1,0,...,3,4}, grid = major]
        \draw[] (0,0) -- (4,4) node[pos=0.98, above]{\tiny1. Winkelhalbierende};
        \draw[dashed] (1.732,3) -- (3,1.732);
        \draw[]  (1.732,3) node[cross] {} node[above right] {\tiny$P(a | b)$};
        \draw[]  (3,1.732) node[cross] {} node[below right] {\tiny$Q(b | a)$};
        \coordinate (A) at (2.4, 2.4);
        \coordinate (B) at (2.35, 2.35);
        \coordinate (C) at (2.38, 2.325);
        \draw pic [draw,angle radius=0.2cm] {angle = C--B--A};
        \node[] at (2.475,2.355) {\tiny$\cdot$};
        \addplot[blue, domain=0:2, samples=50]  {x^2} node[left]{\tiny$G_f$};
        \addplot[red, domain=0:4, samples=100]  {sqrt(x)} node[above]{\tiny$G_{\bar{f}}$};
        \end{axis}
    \end{tikzpicture}
\end{minipage}

\textbf{Bestimmung des Funktionsterms der Umkehrfunktion}

Gegeben ist eine Funktion $f$, die auf der Definitionsmenge $D_f$ umkehrbar ist
\begin{enumerate}
    \item $y = f(x)$ setzen.
    \item Gleichung nach $x$ auflösen.
    \item $x$ und $y$ vertauschen. 
    \item $y$ durch $\bar{f}$ ersetzen.
\end{enumerate}

\textbf{Überprüfung auf Umkehrbarkeit}
\begin{satz}
    Jede streng monotone Funktion ist umkehrbar. Insbesondere ist jede in einem Intervall $I$ differenzierbare Funktion $f$ mit $f^\prime(x) > 0$ bzw. $f^\prime(x) < 0$ für alle $x \in I$ umkehrbar. \footnote{Die strenge Monotonie ist auch erfüllt, wenn für ein $x \in D_f$ (bzw. $x \in I$), das am Rand von $D_f$ (bzw. $I$) liegt, entweder $f^\prime(x) = 0$ gilt oder $f^\prime$ nicht definiert ist.}
\end{satz}

\textit{Beispiel}

$f(x) = \sqrt{4x-2} + 1$\\
$D_f = [0,5 ; \infty[ \ \ \land \ \ W_f = [1 ; \infty[$\\

Überprüfung auf Umkehrbarkeit:
\begin{align*}
    f(x) & = \sqrt{4x-2} + 1 \text{ mit } D_f = [0,5 ; \infty[ && \\
    f^\prime(x) & = \frac{2}{\sqrt{4x - 2}} > 0 \text{ für } x > 0,5 && 
\end{align*}
$\Rightarrow$ Somit ist $f$ auf $D_f$ streng monoton wachsend und damit umkehrbar.

Bestimmen der Umkehrfunktion:
\begin{align*}
    y & = \sqrt{4x-2} + 1 &&\\
    y -1 & = \sqrt{4x-2} &&\\
    (y-1)^2 & = 4x - 2 &&\\
    (y-1)^2 + 2 & = 4x &&\\
    x & = \frac{1}{4} ((y-1)^2 + 2) &&
\end{align*}
Also: $\bar{f}(x) = \frac{1}{4}(y-1)^2 + \frac{1}{2}$ mit $D_{\bar{f}}= W_f$ und $W_{\bar{f}}= D_f$

\section{Die Logarithmusfunktion und ihre Ableitung}

\begin{minipage}{0.35\textwidth}
    Geg.: $f(x) = e^x$\\
    Ges.: Umkehrfunktion $\bar{f}$
    \begin{align*}
        y & = e^x && \\
        \ln{(y)} & = x && \\
        y & = \ln{(x)} &&
    \end{align*}
    Also: $\bar{f}(x) = \ln{(x)}$
\end{minipage}
\begin{minipage}{0.65\textwidth}
    \begin{tikzpicture}[scale=1.5]
        \begin{axis}[clip=true, 
            xmin=-2.5, xmax=4.5, ymin=-2.5, ymax=4.5,
            axis lines = middle, 
            xlabel=$x$, ylabel=$y$,
            xlabel style = {font=\tiny,xshift=0.5ex},
            ylabel style = {font=\tiny,yshift=0.5ex},
            xtick={-2,-1,...,3,4}, xticklabels={-2,-1,...,3,4}, yticklabel style = {font=\tiny,xshift=0.5ex},
            xticklabel style = {font=\tiny,yshift=0.5ex},
            ytick={-2,-1,...,3,4}, yticklabels={-2,-1,...,3,4}, grid = major]
        \addplot[domain=-2:4]{x};
        \addplot[blue, domain=-2:1.45, samples=50]  {e^x} node[left]{\tiny$G_f$};
        \addplot[red, domain=0:4, samples=100]  {ln(x)} node[above]{\tiny$G_{\bar{f}}$};
        \end{axis}
    \end{tikzpicture}
\end{minipage}
\begin{definition}
    Die \textbf{natürliche Logarithmusfunktion} mit $\bar{f}$ mit $\bar{f}(x) = \ln{(x)}$ ist die Umkehrfunktion der natürlichen Exponentialfunktion $f$ mit $f(x) = e^x$. Sie hat die maximale Definitionsmenge $D_{\bar{f}} = ]0;\infty[$ und ihre Wertemenge $W_{\bar{f}} = \mathbb{R}$.
\end{definition}

\textbf{Ableitung der natürlichen Logarithmusfunktion}\\
mit $f(\bar{f}(x)) = x$:
\begin{align*}
    x  & = e^{\bar{f}(x)} && \\
    (x)^\prime & = \left(e^{\bar{f}(x)}\right)^\prime && \\
    1 & = \bar{f}^\prime(x) \cdot e^{\bar{f}(x)} && \\
    1 & = \bar{f}^\prime(x) \cdot x && \\
    \bar{f}^\prime(x) & = \frac{1}{x} &&
\end{align*}

\subsection{$\ln$ Gesetze}
\begin{enumerate}
    \item $\ln{(\frac{1}{x})} = \ln{(1)} - \ln{(x)}$
    \item $\ln{(2x)} = \ln{(2)} + \ln{(x)}$
    \item $\ln{(a^x)} = x \cdot \ln{(a)}$
\end{enumerate}
\noindent\rule{\textwidth}{1pt}
\textit{Beispiel}

$f(x) = 4 \cdot e^{3x-2} + 1$ \\
$D_f = \mathbb{R} \ , \ W_f = \ ] 1 ; \infty [$ \\
$f^\prime(x) = 12 \cdot e^{3x-2} > 0$ Also ist $f$ streng monoton wachsend auf ganz $\mathbb{R}$. Somit ist $f$ umkehrbar.
\begin{align*}
    y & = 4 \cdot e^{3x-2} + 1 && \\
    \frac{y - 1}{4} & = e^{3x-2} && \\
    \ln{\left(\frac{y - 1}{4}\right)} & = 3x-2 && \\
    x & = \frac{1}{3}\left(\ln{\left(\frac{y - 1}{4}\right)} + 2\right) && \\
    y & = \frac{1}{3}\left(\ln{\left(\frac{y - 1}{4}\right)} + 2\right) && \\
    \bar{f}(x) & = \frac{1}{3}\left(\ln{\left(\frac{y - 1}{4}\right)} + 2\right) \qquad D_{\bar{f}} = \ ] 1 ; \infty [ \ \ W_{\bar{f}} = \mathbb{R} &&
\end{align*}

\section{Anwendungen von Exponentialfunktionen}
\subsection{Exponentielle Wachstums- und Zerfallsprozesse}
\begin{definition}
    Exponentielle \textbf{Wachstums- und Zerfallsprozesse} können durch Funktionen $f$ mit $$f(t) = c \cdot a^t \text{ bzw. } f(t) = c \cdot e^{k \cdot t} \text{ mit } k = \ln{(a)} \text{ und } a > 0$$ beschrieben werden. Dabei ist $c \in \mathbb{R}$ der \textbf{Anfangsbestand} $f(0)$ zum Zeitpunkt $t = 0, f(t)$ der Bestand zum Zeitpunkt $t$ und $a$ der \textbf{Wachstumsfaktor}.
    
    Für $0 < a <1$ ist $f$ eine Zerfallsfunktion (negatives Wachstum).\\
    Für $a > 1$ ist $f$ eine Wachstumsfunktion (positives Wachstum). 

    Für $k > 0$ heißt $k$ \textbf{Wachstumskonstante} und $f$ \textbf{Wachstumsfunktion}. \\
    Für $k < 0$ heißt $k$ \textbf{Zerfallskonstante} und $f$ \textbf{Zerfallsfunktion}.
\end{definition}
\noindent\rule{\textwidth}{1pt}
\textit{Beispiel:} Eine Bakterienkultur nimmt mit der Wachstumskonstante $k = 0,03\frac{1}{\text{min}}$ zu. \\
Nach 100 min wurden 800 Bakterien gezählt.\\
a) Wie groß war der Anfangsbestand?\\
b) Wie viele Bakterien werden nach 200 min vorhanden sein?

a) 
\begin{align*}
    f(t) & = c \cdot e^{0,03t}; f(100) = c \cdot e^{0,03 \cdot 100} = c \cdot e^3 && \\
    800 & = c \cdot e^3 \Rightarrow c = \frac{800}{e^3} \approx 39,82 &&
\end{align*}
A.: Der Anfangsbestand betrug ca. 40.

b) \begin{align*}
    f(200) & = 40 \cdot e^{0,03 \cdot 200} = 40 \cdot e^6 \approx 16 137,15 &&
\end{align*}
A.: Nach 200 min werden etwa 16 137 Bakterien vorhanden sein. 
\subsection{Zusammenhang zwischen $p$ und $k$}
Mit der Wachstums- bzw. Zerfallskonstante $k$ verbinden wir keine direkte anschauliche Vorstellung. Meist wird deshalb das Wachstum durch die prozentuale Zunahme $p$ pro Zeitschritt angegeben. \\

Es gilt für die Wachstumskonstante $k$ und die prozentuale Zunahme $p$ pro Zeitschritt: $$\text{Aus: } f(t+1) = a \cdot f(t) = f(t) + \frac{p}{100} \cdot f(t) = \left(1 + \frac{p}{100}\right) \cdot f(t) \text{ folgt } a = \left(1 + \frac{p}{100}\right)$$
Mit $k = \ln{(a)}$ ergibt sich $k = \ln{\left(1 + \frac{p}{100}\right)}$. Nach $p$ auflösen ergibt sich: $p = 100 \cdot \left(e^k - 1\right)$. \\

Analog gilt für die Zerkallskonstante $k$ und die prozentuale Abnahme pro Zeitschritt: $$\text{Aus: } f(t+1) = a \cdot f(t) = f(t) - \frac{p}{100} \cdot f(t) = \left(1 - \frac{p}{100}\right) \cdot f(t) \text{ folgt } a = \left(1 - \frac{p}{100}\right)$$
Mit $k = \ln{(a)}$ ergibt sich $k = \ln{\left(1 - \frac{p}{100}\right)}$. Nach $p$ auflösen ergibt sich: $p = 100 \cdot \left(1 - e^k\right)$.
\subsection{Berechnung der Halbwertzeit $T_H$ bzw. der Verdopplungszeit $T_V$} 
\begin{minipage}[t]{0.49\textwidth}
    Eine für den Zerfallsprozess charakteristische Größe ist die Halbwertszeit, d. h. die Zeit, in der sich der Bestand jeweils halbiert.\\

    Die Halbwertszeit $T_H$ eines Zerfalls berechnen wir folgendermaßen:

    \begin{align*}
        f(t + T_H) & = \frac{1}{2} f(t) \text{ mit } f(t) = c \cdot e^{k \cdot t} && \\
        c \cdot e^{k \cdot (t + T_H)} & = \frac{1}{2} c \cdot e^{k \cdot t} && \\
        c \cdot e^{k \cdot t} \cdot e^{k \cdot T_H} & = \frac{1}{2} c \cdot e^{k \cdot t} && \\
        e^{k \cdot T_H} & = \frac{1}{2} && \\
        k \cdot T_H & = \ln{\left(\frac{1}{2}\right)} && 
    \end{align*}
    Somit ergibt sich für die Halbwertszeit:
    \begin{align*}
        T_H & = \frac{\ln{\left(\frac{1}{2}\right)}}{k} \text{ oder } && \\
        T_H & = -\frac{\ln{(2)}}{k}
    \end{align*}
\end{minipage}
\begin{minipage}{0.02\textwidth}
    \
\end{minipage}
\begin{minipage}[t]{0.49\textwidth}
    Analog gilt für einen Wachstumsprozess und die Verdoppelungszeit $T_V$: \\
    \ \\
    \ \\
    \ \\
    \ \\
    \begin{align*}
        f(t + T_V) & = 2 f(t) \text{ mit } f(t) = c \cdot e^{k \cdot t} && \\
        c \cdot e^{k \cdot (t + T_V)} & = 2 c \cdot e^{k \cdot t} && \\
        c \cdot e^{k \cdot t} \cdot e^{k \cdot T_V} & = 2 c \cdot e^{k \cdot t} && \\
        e^{k \cdot T_V} & = 2 && \\
        k \cdot T_V & = \ln{(2)} && 
    \end{align*}
    \ \\
    \vspace{1.5pt}
    
    Somit ergibt sich für die Verdoppelungszeit:
    \begin{align*}
        T_V & = \frac{\ln{(2)}}{k}&& \\
    \end{align*}
\end{minipage}
\chapter{Integralrechnung}
\section{Rekonstruktion einer Größe}
\textit{Beispiel zu...} \\
\textbf{Orientierter Flächeninhalt}\\
Wir betrachten ein Hybridauto, das zunächst 4 Minuten bergab fährt und dabei gleichmäßig abgebremst wird. Anschließend fährt das Auto 2 Minuten bergauf, wobei es gleichmäßig beschleunigt wird. Während des Abbremsens werden der Batterie konstant 0,175 kWh elektrische Energie pro Minute zugeführt. Bei der Fahrt bergauf werden der Batterie konstant 0,32 kWh elektrische Energie pro Minute entnommen. \\

\begin{minipage}{0.5\textwidth}
    Die Abbildung rechts zeigt diesen \\ 
    Energiefluss für die Batterie, \\
    d.h. die Energiezufuhr (positives Vorzeichen) \\
    bzw. die Energieentnahme (neg. Vorzeichen) \\
    pro Zeiteinheit. \\
    
    Die der Batterie zunächst zugeführte \\
    Energie erhalten wir als Produkt der Zeit, in \\
    der das Auto abgebremst wird, mit dem \\
    Energiezufluss in dieser Zeit, also: $$0,175 \ \frac{\text{kWh}}{\text{min}} \cdot 4 \ \text{min} = 0,7$$
\end{minipage}
\begin{minipage}{0.5\textwidth}
    \begin{tikzpicture}[scale=1.25]
        \begin{axis}[clip=true, 
            xmin=-0.5, xmax=7.5, ymin=-3.5, ymax=2.5,
            axis lines = middle, 
            xlabel=Zeit (in min), ylabel=Energiefluss (in kWh pro min),
            xlabel style = {font=\tiny,xshift=0.5ex},
            ylabel style = {font=\tiny,yshift=0.5ex},
            xtick={0,1,...,6,7}, xticklabels={0,1,...,6,7}, yticklabel style = {font=\tiny,xshift=0.5ex},
            xticklabel style = {font=\tiny,yshift=0.5ex},
            ytick={-3,-2,...,1,2}, yticklabels={{-0,3},{-0,2},{-0,1},0,{0,1},{0,2}}]
        \addplot[thick, domain=0:4]{1.75};
        \addplot[thick, domain=4:6]{-3.2};
        \end{axis}
    \end{tikzpicture}
\end{minipage}

Dieses Produkt können wir in der Abbildung durch ein Rechteck über dem Intervall $[0;4]$ oberhalb der Zeit-Achse veranschaulichen.\\
\ \\

Bei der Fahrt bergauf ist der Energiefluss negativ, da der Batterie Energie entnommen wird. Die Energieentnahme erhalten wir nun als $$0,32 \ \frac{\text{kWh}}{\text{min}} \cdot 4 \ \text{min} = 0,64$$
Sie lässt sich durch ein Rechteck über dem Intervall $[4;6]$ unterhalb der Zeit-Achse veranschaulichen. 

Um deutlich zu machen, dass durch den Flächeninhalt des ersten Rechtecks eine Energiezufuhr und durch den Flächeninhalt des zweiten Rechtecks eine Energieentnahme dargestellt wird, versehen wir die Flächeninhalte mit einem negativen bzw. positiven Vorzeichen und sprechen vom \textbf{orientieten Flächeninhalt}
\begin{itemize}
    \item orientierte Flächeninhalt des ersten Rechtecks: $+0,7$ (ohne Maßeinheiten)
    \item orientierte Flächeninhalt des zweiten Rechtecks: $+0,64$ (ohne Maßeinheiten)
\end{itemize}
Somit veranschaulichen die orientierten Flächeninhalte zusammen direkt (bis auf die Einheiten) den gesamten Energiezufluss bzw. -abfluss über den jeweils betrachteten Zeitintervallen. \\
Die Veränderung des Ladezustandes der Batterie über dem gesamten betrachteten Zeitraum erhalten wir dann als Summe der orientierten Flächeninhalte. \\
Hier ist also $0,7 + (-0,64) = 0,06$ die Veränderung des Ladezustandes im Zeitintervall $[0;6]$.

\section{Das Integral als orientierte Flächeninhalt}
\subsection{Das bestimmte Integral nach Riemann}
\begin{minipage}{0.65\textwidth}
    Bei ungleichmäßig beschleunigten Bewegungen ist es schwierig die Fläche unter der Kurve und damit den Weg exakt zu ermitteln. Die Fläche kann aber näherungsweise mithilfe von Rechteckflächen bestimmt werden. Dies kann z. B. wie in der nebenstehenden Abbildung geschehen.\\

    Riemanns Methode besteht darin, sogenannte \textbf{Unter-} und\textbf{ Obersummen} zu bilden (siehe Abbildungen unten). Dazu teilen wir das gewünschte Intervall in gleich große Abschnitte und bilden jeweils die Summe der Flächeninhalte der Rechtecke unterhalb bzw. oberhalb der Kurve.
\end{minipage}
\begin{minipage}{0.35\textwidth}
    $$f(x)=-0,024 \cdot x^2 + 15$$
    \begin{tikzpicture}[scale=.75]
        \begin{axis}[clip=true, 
            xmin=-0.5, xmax=27.5, ymin=-0.5, ymax=17.5,
            axis lines = middle,
            xlabel=t in s, ylabel=,
            xlabel style = {font=\tiny,xshift=2ex},
            ylabel style = {font=\tiny,yshift=0.5ex},
            xtick={0,5,...,20,25}, xticklabels={0,5,...,20,25}, yticklabel style = {font=\tiny,xshift=0.5ex},
            xticklabel style = {font=\tiny,yshift=0.5ex},
            ytick={0,5,10,15}, yticklabels={0,5,10,15}, grid = major]
            \draw[fill=lightgray!40] (0,0) rectangle (5,14.843);
            \draw[fill=lightgray!40] (5,0) rectangle (10,13.594);
            \draw[fill=lightgray!40] (10,0) rectangle (15,11.094);
            \draw[fill=lightgray!40] (15,0) rectangle (20,7.344);
            \draw[fill=lightgray!40] (20,0) rectangle (25,2.344);
        \addplot[very thick, domain=0:25]{-0.024*x^2 +15};
        \end{axis}
    \end{tikzpicture}
\end{minipage}

\begin{minipage}{0.55\textwidth}
    Der Flächeninhalt A unter der Kurve liegt dann irgendwo dazwischen:
    
    Untersumme < Fläche A < Obersumme.
    
    Je größer die Anzahl n der Teilintervalle ist, desto genauer lässt sich A ermitteln.
    \vspace{-7pt}
    $$\text{Es gilt: }A = \lim_{n\to\infty} U_n = \lim_{n\to\infty} O_n$$
\end{minipage}
\begin{minipage}{0.225\textwidth}
    \begin{tikzpicture}[scale=.5]
        \begin{axis}[clip=false, 
            xmin=-0.5, xmax=27.5, ymin=-0.5, ymax=17.5,
            axis lines = middle,
            xlabel=t in s, ylabel=,
            xlabel style = {font=\tiny,xshift=2ex},
            ylabel style = {font=\tiny,yshift=0.5ex},
            xtick={0,5,...,20,25}, xticklabels={0,5,...,20,25}, yticklabel style = {font=\tiny,xshift=0.5ex},
            xticklabel style = {font=\tiny,yshift=0.5ex},
            ytick={0,5,10,15}, yticklabels={0,5,10,15}, y post scale=1.2, grid = major]
            \draw[fill=lightgray!40] (0,0) rectangle (5,14.4);
            \draw[fill=lightgray!40] (5,0) rectangle (10,12.6);
            \draw[fill=lightgray!40] (10,0) rectangle (15,9.6);
            \draw[fill=lightgray!40] (15,0) rectangle (20,5.4);
            \draw[fill=lightgray!40] (20,0) rectangle (25,0);
            \addplot[very thick, domain=0:25]{-0.024*x^2 +15};
        \end{axis}
    \end{tikzpicture}
    \centering Untersumme $U_5$
\end{minipage}
\begin{minipage}{0.225\textwidth}
    \begin{tikzpicture}[scale=.5]
        \begin{axis}[clip=true, 
            xmin=-0.5, xmax=27.5, ymin=-0.5, ymax=17.5,
            axis lines = middle,
            xlabel=t in s, ylabel=,
            xlabel style = {font=\tiny,xshift=2ex},
            ylabel style = {font=\tiny,yshift=0.5ex},
            xtick={0,5,...,20,25}, xticklabels={0,5,...,20,25}, yticklabel style = {font=\tiny,xshift=0.5ex},
            xticklabel style = {font=\tiny,yshift=0.5ex},
            ytick={0,5,10,15}, yticklabels={0,5,10,15}, y post scale=1.2, grid = major]
            \draw[fill=lightgray!40] (0,0) rectangle (5,15);
            \draw[fill=lightgray!40] (5,0) rectangle (10,14.4);
            \draw[fill=lightgray!40] (10,0) rectangle (15,12.6);
            \draw[fill=lightgray!40] (15,0) rectangle (20,9.6);
            \draw[fill=lightgray!40] (20,0) rectangle (25,5.4);
        \addplot[very thick, domain=0:25]{-0.024*x^2 +15};
        \end{axis}
    \end{tikzpicture}
    \centering Obersumme $O_5$
\end{minipage}

Die auf diese Weise berechnete Fläche nennen wir das \textbf{Riemann-Integral}. \\

\textbf{Merke:}

Die Funktion $f$ sei auf dem Intervall $[a \ ; \ b]$ integrierbar.\\
Die Summe $$\sum^n_{i=1}f(x_i)\cdot h = f(x_1) \cdot h + f(x_2) \cdot h + ... f(x_n) \cdot h + \text{ mit } h = \frac{b - a}{n}$$ sei die \textbf{Zerlegungssumme} von $f$ im Intervall $[a \ ; \ b]$. Der Grenzwert $$\lim_{n \to \infty} S_n = \int_a^bf(x) dx$$ heißt dann das \textbf{bestimmte Integral} der Funktion $f$ in den Grenzen von $a$ bis $b$. \footnote{Diese Zerlegungsfuntionen werden auch Treppenfunktionen genannt.}


\begin{definition}
    Gegeben ist eine Funktion $f$, die auf dem Intervall $[a \ ; \ b]$ integrierbar ist. Das (bestimmte) Integral von $f$ über $[a \ ; \ b]$ ist der orientierte Flächeninhalt, den der Graph von $f$ mit der $x$-Achse zwischen der unteren Grenze $a$ und der oberen Grenze $b$ einschließt.
    
    Wir schreiben hierfür: $$\int_a^bf(x) dx$$
    Hierbei gilt: $$\lim_{n \to \infty} \sum^n_{i = 1}f(x_i) \cdot \Delta x = \int_a^bf(x) dx$$
    \begin{itemize}
        \item Lies: Integral $f(x)$ $dx$ von $a$ bis $b$
        \item $f(x)$ heißt Integrand
        \item $x$ heißt Integrationsvariable
        \item $dx$ gibt die Integrationsvariable an, hierbei steht $d$ für Differential
        \item $a$ heißt untere, $b$ obere Integrationsgrenze
    \end{itemize}
\end{definition}

\section{Bilden von Stammfunktionen}
\textit{Beispiel} \\
Gib eine Funktion an, deren Ableitung eine

\begin{minipage}{0.75\textwidth}
    konstante Funktion, z.B. $f^\prime(x) = 1$, \\
    linerare Funktion, z.B. $f^\prime(x) = x$ bzw. \\
    quadratische Funktion, z.B. $f^\prime(x) = 3x^2$ ist. \\
\end{minipage}
\begin{minipage}{0.25\textwidth}
    $\to \ f(x) = x$ \\
    $\to \ f(x) = \frac{x^2}{2}$ \\
    $\to \ f(x) = x^3$ \\
\end{minipage}

\begin{definition}
    Es sei ein Funktion $f$ gegeben, mit $f(x) = x^n$, so bildet sich die Stammfunktion $F$ folgendermaßen: $$F(x) = \frac{x^{n+1}}{n+1} + C \qquad \footnote{wichtige Ausnahme: $f(x) = \frac{1}{x} \to F(x) = \ln{(|x|)} + C$} $$ 
\end{definition}

\textit{Beispiel: Bestimmen einer speziellen SF (Stammfunktion)} 
\begin{align*}
    f(x) = (x + 2) ^2 \ ; F(0) = 1 \\
    (1) \ F(x) &= \frac{1}{3}(x + 2)^3 + C \\
    (2)\ F(0) &= 1 \\
    1 &= \frac{8}{3} + C \\
    C &= - \frac{5}{3} \quad \quad \Rightarrow F(x) = \frac{1}{3}(x + 2)^3 - \frac{5}{3}
\end{align*}

\subsection{Merkhilfe: Stammfunktionen und Bildungsregeln}

\textbf{Bestimmung von Stammfunktionen}\\
Sind $G$ und $H$ Stammfunktionen von $g$ und $h$, so gelten für die Bildung von Stammfunktionen folgende Regeln:\\
\begin{tabular} { | p{0.2\textwidth} | p{0.2\textwidth} | p{0.2\textwidth} | p{0.2\textwidth} | p{0.2\textwidth} | }
     \hline
       & Potenzregel \linebreak ($n \neq -1$) & Konstanter \linebreak Faktor & Summenregel & Lineare \linebreak Substitution \\
     \hline
     \begin{center} Funktion $f$ mit $f(x) =$ \end{center} & \begin{center} \vspace{5pt}$x^n$ \end{center} & \begin{center} \vspace{5pt}$c \cdot g(x) ; c \in \mathbb{R}$ \end{center} & \begin{center} \vspace{5pt}$g(x) + h(x)$ \end{center}  & \begin{center} \vspace{5pt}$g(m \cdot x + c)$ \end{center} \\
     \hline
     \begin{center} Stammfunktion $F$ mit $F(x) =$ \end{center} & \begin{center} \vspace{5pt}$\frac{1}{n+1}x^{n+1}$ \end{center} & \begin{center} \vspace{5pt}$c \cdot G(x); c \in \mathbb{R}$ \end{center} & \begin{center} \vspace{5pt}$G(x) + H(x)$ \end{center}  & \begin{center} \vspace{5pt}$\frac{1}{m} \cdot G(m \cdot x +c)$ \end{center} \\
    \hline
\end{tabular}
\textbf{Merke:} Die lineare Substitution gilt nur, wenn die innere Funktion eine linere Funktion ist. \\
\textit{Gegenbeispiel:}\\
$f(x) = e^{x^2 + 1}; x \in \mathbb{R}$ \\
Die Funktion $G(x) = \frac{1}{2x} \cdot e^{x^2 + 1}$ ist \textbf{keine Stammkunktion} von $f$, da die Ableitung der Funktion $G$ mithilfe der Quotientenregel sich zu $(G(x))^\prime = \frac{2x \cdot e^{x^2 + 1} \cdot (2x) - e^{x^2 + 1} \cdot 2}{(2x)^2} = e^{x^2 + 1} - \frac{2e^{x^2 + 1}}{4x^2}$ ergibt. Dies ist nicht das Gleiche wie $f(x) = e^{x^2 + 1}$. Somit ist $(G(x))^\prime \neq f(x)$. Daraus folgt, dass $G$ keine Stammfunktion von $f$ ist.

\textbf{\glqq Ableitungskreislauf\grqq \ für Sinus- und Kosinusfunktionen}

\begin{tikzpicture}[node distance=2cm]
    \node (pro1) [process, text width=5cm] {$\sin{(x)}$};
    \node (pro2a) [process, text width=5cm, left of=pro1, xshift = -4cm, yshift= -2cm] {$-\cos{(x)}$};
    \node (pro2b) [process, text width=5cm, right of=pro1, xshift = 4cm, yshift= -2cm] {$\cos{(x)}$};
    \node (pro3) [process, below of=pro1, text width=5cm, yshift= -2cm] {$-\sin{(x)}$};
    
    \begin{scope}[transform canvas={xshift=-0.25em, yshift=0.25em}]
        \draw [arrow, color=red] (pro2a) |- (pro1);
    \end{scope}
    \begin{scope}[transform canvas={xshift=-0.75em, yshift=0.75em}]
        \draw [arrow, color=blue] (pro1) -| (pro2a);
    \end{scope}
    \begin{scope}[transform canvas={xshift=0.25em, yshift=0.25em}]
        \draw [arrow, color=red] (pro1) -| (pro2b);
    \end{scope}
    \begin{scope}[transform canvas={xshift=0.75em, yshift=0.75em}]
        \draw [arrow, color=blue] (pro2b) |- node[above] {Stammfunktion} (pro1);
    \end{scope}
    \begin{scope}[transform canvas={xshift=0.25em, yshift=-0.25em}]
        \draw [arrow, color=red] (pro2b) |- (pro3);
    \end{scope}
    \begin{scope}[transform canvas={xshift=0.75em, yshift=-0.75em}]
        \draw [arrow, color=blue] (pro3) -| (pro2b);
    \end{scope}
    \begin{scope}[transform canvas={xshift=-0.25em, yshift=-0.25em}]
        \draw [arrow, color=red] (pro3) -| node[above right] {Ableitung} (pro2a);
    \end{scope}
    \begin{scope}[transform canvas={xshift=-0.75em, yshift=-0.75em}]
        \draw [arrow, color=blue] (pro2a) |- (pro3);
    \end{scope}
\end{tikzpicture}

\textbf{Ableitung und Stammfunktion zu wichtigen Funktionen}\\
\begin{tabular} { | p{0.087\textwidth} | p{0.087\textwidth} | p{0.087\textwidth} | p{0.087\textwidth} |p{0.087\textwidth} | p{0.087\textwidth} |p{0.087\textwidth} | p{0.087\textwidth} |p{0.087\textwidth} | p{0.087\textwidth} |}
     \hline
     \begin{center} $F(x) =$ \end{center} & \begin{center} $\frac{1}{3}x^3$ \end{center} & \begin{center} $\frac{1}{2}x^2$ \end{center} & \begin{center} $x$ \end{center} & \begin{center}  $\ln{(|x|)}$ \end{center} & \begin{center} $-x^{-1}$ \end{center} &  \vspace{3pt}\centering$x\ln{(x)} -x$  & \begin{center} $e^x$ \end{center} & \begin{center} -$\cos{(x)}$ \end{center} & \begin{center} $\sin{(x)}$  \end{center}  \\
     \hline
     \begin{center} $f(x) =$ \end{center} & \begin{center} $x^2$ \end{center} & \begin{center} $x$ \end{center} & \begin{center} $1$ \end{center} & \begin{center}  $x^{-1}$ \end{center} & \begin{center} $x^{-2}$ \end{center} & \begin{center} $\ln{x}$ \end{center} & \begin{center} $e^x$ \end{center} & \begin{center} $\sin{(x)}$ \end{center} & \begin{center} $\cos{(x)}$  \end{center}  \\
     \hline
     \begin{center} $f^\prime(x) =$ \end{center} & \begin{center} $2x$ \end{center} & \begin{center} $1$ \end{center} & \begin{center} $0$ \end{center} & \begin{center}  $-x^{-2}$ \end{center} & \begin{center} $-2x^{-3}$ \end{center} & \begin{center} $\frac{1}{x}$ \end{center} & \begin{center} $e^x$ \end{center} & \begin{center} $\cos{(x)}$ \end{center} & \begin{center} -$\sin{(x)}$  \end{center}  \\
    \hline
\end{tabular}

\textbf{Hinweis:}\\
Bei allen Stamfunktionen kann beim Funktionsterm ein $+C$ notiert werden. Hierbei ist $C$ eine Konstante.

\section{Der Hauptsatz der Differential- und Integralrechnung (HDI)}

\glqq kurz und knapp:\grqq $$\int_a^bf(x)dx = \left[F(x)\right]^b_a = F(b) - F(a)$$

\textit{Beispiel: Wert des Integrals}
\begin{align*}
    \int_{-3}^{-1}-5dx = \left[-5x\right]_{-3}^{-1} = -5 \cdot (-1) - (-5\cdot(-3)) = 5 - 15 = -10 &&
\end{align*}

\section{Integralfunktion}

\begin{definition}
    Sei $f$ eine Funktion, die im Intervall $I$ integrierbar ist und sei $a \in I$. Die Funktion $J_a$ mit $$J_a(x) = \int_a^xf(t)dt \text{ mit } a \in I$$ heißt \textbf{Integralfunktion von f zur unteren Grenze a}.
\end{definition}

\textbf{Merke:} Eine Integralfunktion hat mindestens eine Nullstelle haben.

\textit{Beispiel:} \\
Die Folgende Funktion $G_a$ soll keine Integralfunktion von g sein: $G_a(x)= 2x^2 -5x + a$\\
Die Werte von a müssen so gewählt werden, dass $G_a$ keine Nullstelle hat.
\begin{align*}
    G_a(x) &= 0 \\
    0 &= 2x^2 -5x + a \\
    x_{1,2} &= \frac{5 \pm \sqrt{25 - 8a}}{4} 
\end{align*}
Wenn $G_a$ keine Nullstelle haben soll, so muss die Diskriminante negativ, also $<0$ sein:
\begin{align*}
    25 - 8a &< 0 \\
    a &> \frac{25}{8}
\end{align*}
Also: $a> \frac{25}{8}$

\section{Stammfunktionen und ihre Graphen}

\textbf{allg. gilt:} $f(x) > 0 \Longleftrightarrow F^\prime(x) > 0$ \\
Damit: F ist streng monoton steigend!

\section{Integral und Flächeninhalt}

offen


\chapter{Funktionen und ihre Graphen}
\section{Strecken, Verschieben und Spiegeln von Graphen}
\textit{Beispiel: Die Veränderung eines Graphen beschreiben}

$f(x) = -(x + 1)^4 + 2$ \\
$g(x) = x^4$ ($g$ hat einen Tiefpunkt $T$ bei $T(0|0)$

Der Graph von $f$ entsteht aus dem Graphen von $g$ durch Spiegelung an der $x$-Achse, Verschiebung um 1 LE\footnote{LE = Längeneinheit, FE = Flächeneinheit, VE = Volumeneinheit} in negative $x$-Richtung und Verschiebung um 2 LE in positive $y$-Richtung.\\
(Der Extrempunkt von $f$ lässt sich somit konstruieren: $H(-1 | 2)$)

\begin{definition}
    Symmetriekriterien: \\
    $f(x) = f(-x) \Longrightarrow \text{Achsensymmetrie zur $y$-Achse}$ \\ 
   $ f(-x) = -f(x) \Longrightarrow \text{Achsensymmetrie zum Ursprung}$ 
\end{definition}

\textit{Beispiel: Symetrie nachweisen}

\begin{minipage}{0.5\textwidth}
    (1) $f(x) = \sqrt{x^2 + 4}$ \\
    $f(-x) = \sqrt{(-x)^2 +4} = \sqrt{x^2 + 4} = f(x)$ \\ $G_f$ ist achsensymmetrisch zur $y$-Achse
\end{minipage}
\begin{minipage}{0.5\textwidth}
    (2) $f(x) = x \cdot e^{-x^2}$ \\
    $f(-x) = (-x) \cdot e^{-(-x)^2} = -x \cdot e^{-x^2} = -f(x)$ \\ $G_f$ ist punktsymmetrisch zum Ursprung
\end{minipage}

\section{Linearfaktordarstellung - mehrfache Nulstellen}

Aufschreib eigenständig - nachtragen!!

\section{Lösen von Gleichungen}

\textit{Beispiele}\\
(1) Zum lösen von Wurzelgleichungen wird immer folgendes Schema angewendet: \\ \textbf{Wurzelterm isolieren und dann Wurzel auflösen}
\begin{align*}
    \sqrt{x + 1} +1 &= 0 \\
    \sqrt{x + 1} & = -1 \\
    x + 1 &= 1 \\
    x &= 0 
\end{align*}

\textbf{Achtung!} Bei Wurzelgleichungen ist eine Probe nachher \textbf{immer} obligatorisch, da das quadrieren einer Wurzel keine Äquivalenzumformung ist.

\textbf{ENTWEDER: }

\textit{Probe:}\\
$\sqrt{0 + 1} + 1 = 2 \neq 0$ \\
Damit: $\mathbb{L} = \{\}$

\textbf{ODER:}

$\sqrt{x + 1} \neq -1 \ \forall x \in \mathbb{R}$ \\
Damit: $\mathbb{L} = \{\}$

(2) Bruchgleichungen lassen sich meist durch einfaches umformen lösen.
\begin{align*}
    \frac{x}{x - 1} &= \frac{3}{4} \qquad  \qquad |\cdot (x - 1); \ \cdot 4  \\
    4x &= 3x - 3 \\
    x &= -3
\end{align*}
Damit: $\mathbb{L} = \{-3\}$

\newpage 

(3) Bei Betragsgleichungen ist immer eine Fallunterscheidung und eine Probe notwendig. Es ist notwendig, bei einem beliebigem Fall, den Fall der Gleichheit einzuschließen. 
\begin{align*}
    |x - 4| &= 2x - 11 \\
\end{align*}
\begin{minipage}[t]{0.5\textwidth}
    1. Fall: $|x - 4| \geq 0$
    \begin{align*}
        x - 4 &= 3x - 11 \\
        x_1 &= 7
    \end{align*}
    \textit{Probe:} \\
    für $x_1 = 7$ gilt: 
    \begin{align*}
        |7 - 4| &= 2 \cdot 7 - 11 \\
        3 &= 3
    \end{align*}
\end{minipage}
\begin{minipage}[t]{0.5\textwidth}
    2. Fall: $|x - 4| < 0$
    \begin{align*}
        -(x - 4) &= 2x - 11 \\
        -x + 4 &= 2x - 11 \\
        x_2 &= 5
    \end{align*}
    \textit{Probe:}\\
    für $x_2 = 5$ gilt:
    \begin{align*}
        |5 - 4| &= 2 \cdot 5 - 11 \\
        1 &\neq -1
    \end{align*}
\end{minipage}

Damit: $\mathbb{L} = \{7\}$

\section{Trigonometrische Funktionen}

\begin{minipage}{0.55\textwidth}
    \begin{definition}
    Die Sinusfunktion $f$ mit $$f(x) = a \cdot \sin(b \cdot x) \text{ mit } a \neq 0 \text{ und } b > 0$$ hat die Amplitude $|a|$ und die Periode $p = \frac{2\pi}{b}$. \\
\end{definition}
\end{minipage}
\begin{minipage}{0.45\textwidth}
    \begin{tikzpicture}[scale=1]
        \begin{axis}[clip=true, 
            ymin=-2, ymax=2, xmin=-.5, xmax=7.5,
            axis lines = middle, 
            xlabel=$x$, ylabel=$y$,
            xlabel style = {font=\tiny,xshift=0.5ex},
            ylabel style = {font=\tiny,yshift=0.5ex},
            ytick={-2,-1,...,1,2}, yticklabels={-2,-1,...,1,2}, ticklabel style = {font=\tiny,xshift=0.5ex},
            xticklabel style = {font=\tiny,yshift=0.5ex},
            xtick={1,2,...,6,7}, xticklabels={1,2,...,6,7}, grid = major, width= 10cm, height = 5cm]
        \addplot[samples= 100, domain=0:6.282]{sin(57.25*x)} node[right, pos=0.4]{\tiny$G_f \text{ mit } f(x) = \sin{(x)}$};
        \end{axis}
    \end{tikzpicture}
\end{minipage}
\begin{definition}
     Der Graph der Sinusfunktion $f$ mit $$f(x) = a \cdot \sin(b(x - c)) + d$$ ist zusätzlich um $c$ LE in positive $x$-Richtung und um $d$ LE in positive $y$-Richtung verschoben. \\ analoge Betrachtung: $f(x) = \cos(x)$
\end{definition}


\end{document}
