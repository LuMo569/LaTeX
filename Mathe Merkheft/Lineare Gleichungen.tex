\chapter{Lineare Gleichungen}
\section{Der Gauß-Algorithmus}

Ziel: LGS\footnote{= lineares Gleichungssystem, meistens drei Gleichungen mit drei Unbekannten} mithilfe von Äquivalenzumformungen auf Stufen-/Dreiecksform bringen\\

\begin{minipage}{0.6\textwidth}
    $\begin{gmatrix}
    3x_1 & + & 6x_2 & - & 2x_3 & = & -4 \\
    3x_1 & + & 2x_2 & + & x_3 & = & 0 \\
    \frac{3}{2}x_1 & + & 5x_2 & - & 5x_3 & = & -9 \\
    \rowops
    \mult{0}{(-1)}\add{0}{1}
    \mult{2}{(-2)}\add{0}{2}
    \end{gmatrix}$
    
    \par\noindent\rule{0.65\textwidth}{0.4pt}
    
    $\begin{gmatrix}
    3x_1 & + & 6x_2 & - & 2x_3 & = & -4 \\
    & - & 4x_2 & + & 3x_3 & = & 4 \\
    & - & 4x_2 & + & 8x_3 & = & 14 \\
    \rowops
    \mult{1}{(-1)}\add{1}{2}
    \end{gmatrix}$
    
    \par\noindent\rule{0.65\textwidth}{0.4pt}
    
    $\begin{gmatrix}
    3x_1 & + & 6x_2 & - & 2x_3 & = & -4 \\
    & - & 4x_2 & + & 3x_3 & = & 4 \\
    & & & & 5x_3 & = & 10\\
    \end{gmatrix}$
    
    \par\noindent\rule{0.65\textwidth}{0.4pt}
    
    $\begin{gmatrix}
    \text{Also:} & \quad & \ & \ & \ & x_3 & = & 3 \\
    & \quad & \ & \ & \ & x_2 & = & 0,5 \\
    & \quad & \ & \ & \ & x_1 & = & -1 
\end{gmatrix}$

\ \newline
Damit: $\mathbb{L}$ = \{$(-1; \ 0,5; \ 2)$\}

\end{minipage}
\begin{minipage}{0.4\textwidth}
    \textbf{Matrixform\footnotemark:}

\begin{equation*}
    \begin{aligned}
        \begin{gmatrix}[p]
        3 \ & 6 \ & -2 & \BAR & -4 \\
        3 \ & 2 \ & 1 & \BAR & 0 \\
        \frac{3}{2} \ & 5 \ & -5 & \BAR & -9 
        \rowops
        \mult{0}{(-1)}\add{0}{1}
        \mult{2}{(-2)}\add{0}{2}
        \end{gmatrix}
        
        \par\noindent\rule{0.7\textwidth}{0.4pt}
        
        \begin{gmatrix}[p]
        3 & 6 & -2 & \BAR & -4 \\
        0 & -4 & 3 & \BAR & 4 \\
        0 & -4 & 8 & \BAR & 14 
        \rowops
        \mult{1}{(-1)}\add{1}{2}
        \end{gmatrix}
        
        \par\noindent\rule{0.7\textwidth}{0.4pt}
        
        \begin{gmatrix}[p]
        3 & 6 & -2 & \BAR & -4 \\
        0 & -4 & 3 & \BAR & 4 \\
        0 & 0  & 5 & \BAR & 10
        \end{gmatrix}
    \end{aligned}
\end{equation*}

\end{minipage}

\footnotetext{In der Matixform werden nur die Koeffizienten geschrieben, das spart Zeit und man verwechselt keine Variablen}

\pagebreak

\section{Lösungsmengen linearer Gleichungssysteme}

\begin{satz}
    Ein LGS hat entweder \textbf{genau eine} oder \textbf{keine} Lösung oder \textbf{unendlich viele} Lösungen.
\end{satz}
\noindent\rule{\textwidth}{1pt}

\textit{Beispiele:}

\begin{minipage}[t]{0.5\textwidth}
    (1) eine Lösung
    \begin{equation*}
        \begin{aligned}
            \begin{gmatrix}[p]
            1  & 2  & 1 & \BAR & 9 \\
            -2  & -1  & 5 & \BAR & 5 \\
            1  & -1  & 3 & \BAR & 4 
            \rowops
            \mult{0}{\cdot2}\add{0}{1}
            \mult{0}{(-1)}\add{0}{1}
            
            \end{gmatrix}
            
            \par\noindent\rule{0.5\textwidth}{0.4pt}
            
            \begin{gmatrix}[p]
            1  & 2  & 1 & \BAR & 9 \\
            0 & 3 & 7 & \BAR & 23 \\
            0 & -3 & 2 & \BAR & -5
            \rowops
            \add{1}{2}
            \end{gmatrix}
            
            \par\noindent\rule{0.5\textwidth}{0.4pt}
            
            \begin{gmatrix}[p]
            1  & 2  & 1 & \BAR & 9 \\
            0 & 3 & 7 & \BAR & 23 \\
            0 & 0  & 9 & \BAR & 18
            \end{gmatrix}
        \end{aligned}
    \end{equation*}
    
    Also: $x_3 = 2, \ x_2 = 3, \ x_1 = 1$ \\
    $\mathbb{L}$ = \{$(1; \ 3; \ 2)$\} \\

    (2) keine Lösung
    \begin{equation*}
        \begin{aligned}
            \begin{gmatrix}[p]
            1  & 2  & 3 & \BAR & 4 \\
            1  & 2  & -2 & \BAR & 0 \\
            -3  & -6  & 6 & \BAR & -1 
            \rowops
            \mult{0}{(-1)}\add{0}{1}
            \mult{0}{(3)}\add{0}{2}
            
            \end{gmatrix}
            
            \par\noindent\rule{0.5\textwidth}{0.4pt}
            
            \begin{gmatrix}[p]
            1  & 2  & 3 & \BAR & 4 \\
            0 & 0 & -5 & \BAR & -4 \\
            0 & 0 & 15 & \BAR & 11
            \rowops
            \end{gmatrix}
        \end{aligned}
    \end{equation*}
    
    $x_3$ hat zwei verschiedene Lösungen \\ $\rightarrow$ Widerspruch\\
    $\Rightarrow$ keine Lösung bzw. $\mathbb{L}$ = \{\} \\
    
\end{minipage}
\begin{minipage}[t]{0.5\textwidth}
    (3) unendlich viele Lösungen
    \begin{equation*}
    \begin{aligned}
        \begin{gmatrix}[p]
        2 \ & 6  & -3 & \BAR & -6 \\
        4 \ & 3  & 3 & \BAR & 6 \\
        4  & -3  & 9 & \BAR & 18 
        \rowops
        \mult{0}{(-2)}\add{0}{1}\add{0}{2}
            
        \end{gmatrix}
            
        \par\noindent\rule{0.5\textwidth}{0.4pt}
            
        \begin{gmatrix}[p]
        2  & 6  & -3 & \BAR & -6 \\
        0 & -9 & 9 & \BAR & 4 \\
        0 & -15 & 15 & \BAR & 30 
        \rowops
        \mult{1}{(-5)}\mult{2}{\cdot3}\add{1}{2}
        \end{gmatrix}
            
        \par\noindent\rule{0.5\textwidth}{0.4pt}
            
        \begin{gmatrix}[p]
        2 & 6 & -3 & \BAR & -6 \\
        0 & -9 & 9 & \BAR & 4 \\
        0 & 0  & 0 & \BAR & 0
        \end{gmatrix}
    \end{aligned}
    \end{equation*}
    
Also: $x_3 = t$ \quad , $t \in \mathbb{R}$ \\
Damit: $x_2 = -2-t$, $x_1 = 3 + \frac{9}{2}t$
    
$\mathbb{L}$ = $\left\{\left(3 + \frac{9}{2}t; \ -2-t; \ t\right) \middle| t \in \mathbb{R}\right\}$

\end{minipage}

\ \\
\ \\
\ \\


\section{LGS mit Parametern auf der rechten Seite}

\textit{Beispiele:}

\begin{minipage}[t]{0.5\textwidth}
    (1) Bestimmen sie den Wert von $r$ so, \\ dass das LGS die die angegebene Lösung \\ besitzt. \\
    \begin{equation*}
        \begin{aligned}
            \begin{gmatrix}[p]
            2 & -2 & 1 & \BAR & -3 \\
            3 & -1 & 1 & \BAR & r-2 \\
            4 & 1 & -1 & \BAR & 3,5r
    \end{gmatrix}
        \end{aligned}
    \end{equation*}
    
    Lösung: $(1; \ 2; \ -1)$ \\
    
    (1) $-3 = -3$\\
    (2) $0 = r-2 \Leftrightarrow r = 2$\\
    (3) $7 = 3,5r \ \Leftrightarrow r = 2$
\end{minipage}
\begin{minipage}[t]{0.5\textwidth}
    (2) Untersuchen Sie, ob es Werte von $r$ gibt, sodass das LGS keine oder unendlich viele Lösungen besitzt. Geben Sie im Fall unendlich vieler Lösungen die Lösungsmenge an. \\
    \begin{equation*}
    \begin{aligned}
        \begin{gmatrix}[p]
        2 & 1 & 1 & \BAR & 1 \\
        0 & -1 & 1 & \BAR & 2 \\
        0 & 3 & -3 & \BAR & r
        \rowops
        \mult{1}{\cdot 3}\add{1}{2}
        \end{gmatrix}
        
        \par\noindent\rule{0.5\textwidth}{0.4pt}
                
        \begin{gmatrix}[p]
            2 & 1 & 1 & \BAR & 1 \\
            0 & -1 & 1 & \BAR & 2 \\
            0 & 0 & 0 & \BAR & r + 6 
        \end{gmatrix}
    \end{aligned}
    \end{equation*}

    Aus $0 = r + 6$ folgt: \\
    für $r \neq 6$ besitzt das LGS keine Lösung, also $\mathbb{L} = \{\}$ \\
    für $r = -6$ besitzt das LGS unendlich viele Lösungen:\\
    
    Setze $x_3 = t \qquad ,t \in \mathbb{R}$ \\
    Damit= $\mathbb{L}_{-6} = \left\{\left(-t + \frac{3}{2}; \ t - 2 ; \ t\right) \middle| t \in \mathbb{R}\right\}$
\end{minipage}
